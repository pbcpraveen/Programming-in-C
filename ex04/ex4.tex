% Created 2018-04-15 Sun 01:29
\documentclass[11pt]{article}
\usepackage[utf8]{inputenc}
\usepackage[T1]{fontenc}
\usepackage{fixltx2e}
\usepackage{graphicx}
\usepackage{longtable}
\usepackage{float}
\usepackage{wrapfig}
\usepackage{rotating}
\usepackage[normalem]{ulem}
\usepackage{amsmath}
\usepackage{textcomp}
\usepackage{marvosym}
\usepackage{wasysym}
\usepackage{amssymb}
\usepackage{hyperref}
\tolerance=1000
\usepackage{palatino}
\usepackage[top=1in,bottom=1.25in,left=1.25in,right=1.25in]{geometry}
\usepackage{setspace}
\setcounter{secnumdepth}{1}
\author{PRAVEEN KUMAR R}
\date{27/2/18}
\title{Exercise 4: Iterative statements}
\hypersetup{
  pdfkeywords={},
  pdfsubject={},
  pdfcreator={Emacs 24.5.1 (Org mode 8.2.10)}}
\begin{document}

\maketitle
\begin{center}
\begin{tabular}{lr}
Assignment & 4\\
Reg No & 312217104114\\
Name & PRAVEEN KUMAR R\\
Grade & \\
Date & 27-02-2018\\
\end{tabular}
\end{center}
\begin{export}
\linespread{1.2}
\end{export}


\section{Indenting using iteration}
\label{sec-1}

\subsection*{Problem description:}
\label{sec-1-1}
Define a function \texttt{indent()} to print n times the pattern 
\texttt{|-{}-} in a line where n is given as a parameter to the function

\subsection*{Specifications:}
\label{sec-1-2}
function called \texttt{indent()} to print the pattern \texttt{|-{}-}
n times followed by the key n
\texttt{input}: key ,value from the standard input
\texttt{output}: \texttt{|-{}-|-{}-...} for n times \texttt{key}

\subsection*{Program code:}
\label{sec-1-3}

\begin{verbatim}
#include<stdio.h>
void indent(int level,int key)
{
  for(int i=1;i<=level;i++)
    printf("|--");
  printf("%d\n",key);
}
int main()
{
  int a,a1;
  while( scanf("%d%d",&a,&a1)!=EOF)
    {
       indent(a,a1);
    }
}
\end{verbatim}
\subsection*{Test}
\label{sec-1-4}
\subsubsection*{Input}
\label{sec-1-4-1}
\begin{verbatim}
2 10
3 45
6 60
7 7
\end{verbatim}

\subsubsection*{Output}
\label{sec-1-4-2}
\begin{verbatim}
|--|--10
|--|--|--45   
|--|--|--|--|--|--60
|--|--|--|--|--|--|--7
\end{verbatim}

\section{Length of the array}
\label{sec-2}

\subsection*{Problem description:}
\label{sec-2-1}
define a function \texttt{array\_len()} to find the length of 
the array terminated by -1
\subsection*{Specifications:}
\label{sec-2-2}
function called \texttt{array\_len()} to find out the length of an array
terminated by -1
\texttt{input}: an array terminated by -1
\texttt{output}: length of the array

\subsection*{Program code:}
\label{sec-2-3}

\begin{verbatim}
#include<stdio.h>
int arraylen(int a[])
{
  int c=1,i=0;
  while(a[i]!=-1)
    {
      c+=1;
      i+=1;
    }
  return c;
}
int main()
{
  int a[100];
  for(int i =0;;i++)
    {
      scanf("%d",&a[i]);
      if(a[i]==-1)
	break;
    }

  printf("%d\n",arraylen(a));
}
\end{verbatim}
\subsection*{Test}
\label{sec-2-4}
\subsubsection*{Input}
\label{sec-2-4-1}
4 63 19 4 3 6 8 0 2 8 -1
\subsubsection*{Output}
\label{sec-2-4-2}
\begin{verbatim}
8
\end{verbatim}

\subsection*{Sample input:}
\label{sec-2-5}
4 3 6 8 0 2 8 -1

\subsection*{Sample output:}
\label{sec-2-6}
8

\section{Print a subarray}
\label{sec-3}

\subsection*{Problem description}
\label{sec-3-1}
write a function \texttt{print\_array()} that prints a subarray \texttt{a[low:high]} that is 
items from low to high

\subsection*{Specifications}
\label{sec-3-2}
function called \texttt{print\_array()} to print the subarray from low to high
\texttt{input}: array,lower bound,upper bound
\texttt{output}: subarray \texttt{a[low:high]}

\subsection*{Program code}
\label{sec-3-3}

\begin{verbatim}
#include<stdio.h>
void print_array(int a[],int l,int h)
{
  for(int i=l;i<h;i++)
    printf("%d%c",a[i],i-1<h?',':' ');
}
int main()
{
  int n,low,high,a[100];
  scanf("%d",&n);
  for(int i=0;i<n;i++)
    scanf("%d",&a[i]);
  scanf("%d%d",&low,&high);
  print_array(a,low,high);
}
\end{verbatim}
\subsection*{Test}
\label{sec-3-4}
\subsubsection*{Input}
\label{sec-3-4-1}
\begin{verbatim}
10 23 45 22 11 233 45453  221 344 4565 223  
2 8
\end{verbatim}
\subsubsection*{Output}
\label{sec-3-4-2}
\begin{center}
\begin{tabular}{rrrrrr}
22 & 11 & 233 & 45453 & 221 & 344\\
\end{tabular}
\end{center}

\subsection*{Sample input}
\label{sec-3-5}
1 2 3 4 5 6 7 
3 5

\subsection*{Sample output}
\label{sec-3-6}
4 5

\section{Sum,mean,variance}
\label{sec-4}
\subsection*{Program description}
\label{sec-4-1}
Read an array of numbers from standard input.
Define \texttt{sum()} , \texttt{variance()} and \texttt{mean()}, to calculate
the sum ,variance and mean of the given set of numbers.
Test these functions from the \texttt{main()}.
\subsection*{Specification}
\label{sec-4-2}
\begin{itemize}
\item \texttt{sum()}
-- Input: a[0:n], anarray of integers and n, number of elements 
   in the array.
-- Output: sum of the set of numbers.
\item \texttt{variance()}
-- Input: a[0:n], anarray of integers and n, number of elements 
   in the array.
-- Output: variance of the set of numbers.
\item \texttt{mean()}
-- Input: a[0:n], anarray of integers and n, number of elements 
   in the array.
-- Output: mean of the set of numbers.
\end{itemize}
\subsection*{Algorithm}
\label{sec-4-3}
\begin{verbatim}
def sum(a,l,h):
   s=0
   for i in range(l,h):
      s+=a[i]
   return s
def mean(a,l,h):
   return sum(a,l,h)/(1.0*(h-l))
def variance(a,l,h):
   m=mean(a,l,h),s=0
   for i in range(l,h):
      s+=(a[i]-m)^2
   return s/(h-l)
def count(a,l,h):
   m=mean(a,l,h)
   s=0
   for i in range(l,h):
      if a[i]>m:
         s++
   return s
\end{verbatim}
\subsection*{Program}
\label{sec-4-4}
\begin{verbatim}
#include<stdio.h>
int sum(int a[],int n);
float variance(int a[],int n);
float mean(int a[],int n);
int sum(int a[],int n)
{
  int sum=0;
  for(int i=0;i<n;i++)
    {
      sum+=a[i];
    }
  return sum;
}
float mean(int a[],int n)
{
  return sum(a,n)/((float)(n));
}
float variance(int a[],int n)
{
  float s=mean(a,n);
  float k=0;
  for(int i=0;i<n;i++)
    {
      float p=a[i]-s;
      k=k+p*p;
    }
  return k/n;
}
int main()
{
  int a[100],n;
  scanf("%d",&n);
  for(int i=0;i<n;i++)
    scanf("%d",&a[i]);
  printf("sum=%d\nmean=%f\nvariance=%f",sum(a,n),mean(a,n),variance(a,n));
}
\end{verbatim}
\subsection*{Test}
\label{sec-4-5}
\subsubsection*{Input}
\label{sec-4-5-1}
10
46 38 283 54 23 239 46 224  24 245 
\subsubsection*{Output}
\label{sec-4-5-2}
\begin{center}
\begin{tabular}{l}
sum=1222\\
mean=122.199997\\
variance=10777.959961\\
\end{tabular}
\end{center}

\section{Prime numbers}
\label{sec-5}

\subsection*{Problem description:}
\label{sec-5-1}
write a boolean function \texttt{is\_prime()} to check whether a number is prime or not 
and prints true or false

\subsection*{Specifications:}
\label{sec-5-2}
function \texttt{is\_prime()} to check whether a number is prime or not 
and check the first 100 integers
\texttt{input}: an integer n
\texttt{output}: true if the number is prime , 
          flase if the number is not prime

\subsection*{Algorithm}
\label{sec-5-3}
\begin{verbatim}
def is_prime(a):
   i=2,f=1
   while i<a/2:
      if a%i==0:
         f=0
         break
      i++
   return f
\end{verbatim}
\subsection*{program code:}
\label{sec-5-4}

\begin{verbatim}
#include<stdio.h>
#include<stdbool.h>
bool is_prime(int n)
{
  int flag=1;
  for(int i=2;i<n;i++)
    {
      if(n%i==0)
	flag =0;
    }
  if(flag==1)
    return true;
  else
    return false;
}
int main()
{
  int a[100];
  int n;
  for(n=0;scanf("%d",&a[n])!=EOF;n++);
  for(int i=0;i<n;i++)
    {
      printf("%d :",a[i]);
      if(is_prime(a[i]))
	printf("true");
      else
	printf("false");
      printf("\n");
    }
}
\end{verbatim}

\begin{center}
\begin{tabular}{rl}
23 & :true\\
45 & :false\\
75 & :false\\
32 & :false\\
234 & :false\\
24 & :false\\
111 & :false\\
3545 & :false\\
\end{tabular}
\end{center}

\subsection*{Test}
\label{sec-5-5}
\subsubsection*{Input}
\label{sec-5-5-1}
23 45 75 32 234 24 111 3545 
\subsubsection*{Output}
\label{sec-5-5-2}
\begin{center}
\begin{tabular}{rl}
23 & :true\\
45 & :false\\
75 & :false\\
32 & :false\\
234 & :false\\
24 & :false\\
111 & :false\\
3545 & :false\\
\end{tabular}
\end{center}

\section{Linear search}
\label{sec-6}

\subsection*{Problem description:}
\label{sec-6-1}
define a function \texttt{linear\_search()} to search a target in an array and return the index if the item is present 
or else return an invalid index

\subsection*{Algorithmic process:}
\label{sec-6-2}
\begin{itemize}
\item compare each item in the array with the target element
\item If they are equal then return the index thereby breaking the loop.
\item If the target is not in the array then the loop will end when control variable reaches an invalid index
\end{itemize}
then the function will return an invalid index

\subsection*{Algorithm}
\label{sec-6-3}
\begin{verbatim}
def linear_search(a,n,t):
   for i in range(n):
      if a[i]==t:
         break
   return i
def linear_search_n(a,n,t):
   i=0
   while i<n and a[i]!=t:
      i=i+1
   return i
def binary_search(a,n,t):
   l=0,u=n-1,f=0,m
   while l<=u and f=0:
      m=(l+u)/2
      if t==a[m]:
         f=m
      elif a[m]>t:
         u=m-1
      else:
         l=m+1
   if f==0:
      return -1
   return f
\end{verbatim}
\subsection*{Specifications:}
\label{sec-6-4}
\texttt{linear\_search()}: to search for a target in an array and return the index of the array if present or 
invalid number if the target is not present
\begin{itemize}
\item \texttt{input}: a[0:n],an integer array, n ,target
\item \texttt{output}: index of target or the length of the array(invalid)
\end{itemize}

\subsection*{Program code}
\label{sec-6-5}
\begin{verbatim}
#include<stdio.h>
int linear_search(int a[],int n,int t)
{
  int i;
  for(i=0;i<n;i++)
  {
    if(a[i]==t)
      break;
  }
  return i;
}
int main()
{
  int n,t,a[100];
  scanf("%d",&n);
  for(int i=0;i<n;i++)
    scanf("%d",&a[i]);
  scanf("%d",&t);
  printf("%d\n",linear_search(a,n,t));
}
\end{verbatim}
\subsection*{Test}
\label{sec-6-6}
\subsubsection*{Input}
\label{sec-6-6-1}
\begin{verbatim}
7
3 42 4 42 22 112  44
112
\end{verbatim}
\subsubsection*{Output}
\label{sec-6-6-2}
\begin{verbatim}
5
\end{verbatim}

\section{Minimum}
\label{sec-7}

\subsection*{Problem description:}
\label{sec-7-1}
write a function \texttt{min()} that returns the index of the 
smallest item in the array

\subsection*{Algorithmic process:}
\label{sec-7-2}
\begin{itemize}
\item Assume first element of the array as the minimum.
\item Compare each item in the array with the minimum assumed .
\item if they are smaller than the minimum then assign that item as the minimum and continue the loop.
\item After each iteration the minimum variable will have the minimum value of the sub-array a[0,i].
\item After loop ends the variable will have the minimum number of the array
\end{itemize}
\subsection*{Algorithm}
\label{sec-7-3}
\begin{verbatim}
def min(a,l,h):
   m=l
   for i in range(l+1,h):
      if a[i]<a[m]:
         m=i
   return m
\end{verbatim}
\subsection*{Specifications:}
\label{sec-7-4}
\texttt{min()} to return the index of the minimum item in the array
\texttt{input}: a[0:n], an array of integers and n
\texttt{output}: i such that a[i]<=a[0:n]
\subsection*{Program code:}
\label{sec-7-5}

\begin{verbatim}
#include<stdio.h>
int min(int a[],int n)
{
  int m=0,i;
  for(i=0;i<n;i++)
  {
    if(a[i]<a[m])
      m=i;
  }
  return m;
}
int main()
{
  int n,a[100];
  scanf("%d",&n);
  for(int i=0;i<n;i++)
    scanf("%d",&a[i]);
  printf("%d\n",a[min(a,n)]);
}
\end{verbatim}
\subsection*{Test}
\label{sec-7-6}
\subsubsection*{Input}
\label{sec-7-6-1}
\begin{verbatim}
10
33 5432 4254 545 211 45 125 44 533 65672
\end{verbatim}
\subsubsection*{Output}
\label{sec-7-6-2}
\begin{verbatim}
33
\end{verbatim}

\section{Armstrong number}
\label{sec-8}
\begin{enumerate}
\item Define a function \texttt{int to\_digits(n, s)} to convert an integer
to a string of single digit numbers. For example, it converts
371 to [3,7,1]. The function has two outputs:
\begin{enumerate}
\item \texttt{s}, an array of single digit numbers, which is passed as a
parameter, and
\item the number of single digits, which is returned as a value.
\end{enumerate}
Test the function from \texttt{main()}.
\item Define a function \texttt{cube(x)} that returns $x^3$.
\item Write a function \texttt{is\_armstrong(n)} that tests whether the
integer \texttt{n} is an Armstrong number. An Armstrong number is
equal to the sum of cubes of its digits. Test the function to
find out all the Armstrong numbers from 0 to 500.
\end{enumerate}
\subsection*{Specification}
\label{sec-8-1}
3 functions \texttt{to\_digits()}, which gets the number \texttt{n} and array \texttt{a[]}
as input, stores each digit in the array and returns number of digits,
\texttt{cube()}, which finds the cube of a number, and \texttt{is\_armstrong()}, which
gets the number, each individual digit and its length as input and checks
if a number is armstrong or not.
\subsection*{Prototype}
\label{sec-8-2}
\begin{verbatim}
int to_digits(int n, int s[])
int cube(int n)
int is_armstrong(int n, int s[], int b)
\end{verbatim}
\subsection*{Program Design}
\label{sec-8-3}
The program consists of 3 functions \texttt{to\_digits(int n, int s[])} which finds
number of digits and stores them in an array, \texttt{cube(int n)} which finds cube 
of a number, \texttt{is\_armstrong(int n, int s[], int b)} which checks if a number is
armstrong or not, and \texttt{main()}, which gets the input from \texttt{stdin}, calls the
functions and prints the result on \texttt{stdout}.
\subsection*{Algorithm}
\label{sec-8-4}
\begin{verbatim}
def to_digits(n,s):
   i=0
   while n!=0:
      s[i]=n%10
      n/=10
      i+=1
   return i
def cube(n):
   return n*n*n
def is_armstrong(n,s,b):
   a=0
   for i in range(b):
      a+=cube(s[i])
   if n==a:
      return 1
   return 0
\end{verbatim}
\subsection*{Source Code}
\label{sec-8-5}
\begin{verbatim}
#include<stdio.h>
int to_digits(int n, int s[]){
  int i=0;
  while(n!=0){
    s[i]=n%10;
    n/=10;
    i++;
  }
  return i;
}
int cube(int n){
  return n*n*n;
}
int is_armstrong(int n, int s[], int b){
  int a=0;
  for(int i=0;i<b;i++){
    a+=cube(s[i]);
  }
  if(n==a){
    return 1;
  }
  return 0;
}
int main(){
  int n,s[30],f,a;
  scanf("%d",&n);
  a=to_digits(n,s);
  f=is_armstrong(n,s,a);
  if(f==1){
    printf("Armstrong");
  }
  else{
    printf("Not Armstrong");
  }
}
\end{verbatim}
\subsection*{Test}
\label{sec-8-6}
\subsubsection*{Input}
\label{sec-8-6-1}
\begin{verbatim}
1634
345
\end{verbatim}
\subsubsection*{Output}
\label{sec-8-6-2}
\begin{center}
\begin{tabular}{l}
Armstrong\\
Not Armstrong\\
\end{tabular}
\end{center}
% Emacs 24.5.1 (Org mode 8.2.10)
\end{document}
