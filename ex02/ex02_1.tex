% Created 2018-04-15 Sun 01:42
\documentclass[11pt]{article}
\usepackage[utf8]{inputenc}
\usepackage[T1]{fontenc}
\usepackage{fixltx2e}
\usepackage{graphicx}
\usepackage{longtable}
\usepackage{float}
\usepackage{wrapfig}
\usepackage{rotating}
\usepackage[normalem]{ulem}
\usepackage{amsmath}
\usepackage{textcomp}
\usepackage{marvosym}
\usepackage{wasysym}
\usepackage{amssymb}
\usepackage{hyperref}
\tolerance=1000
\usepackage{palatino}
\usepackage[top=1in,bottom=1.25in,left=1.25in,right=1.25in]{geometry}
\usepackage{setspace}
\setcounter{secnumdepth}{1}
\author{PRAVEEN KUMAR R}
\date{5/2/18}
\title{Exercise 2: Expresions,Variables,Assignments}
\hypersetup{
  pdfkeywords={},
  pdfsubject={},
  pdfcreator={Emacs 24.5.1 (Org mode 8.2.10)}}
\begin{document}

\maketitle
\begin{export}
\linespread{1.2}
\end{export}
\begin{center}
\begin{tabular}{lr}
Assignment & 1\\
Reg No & 312217104114\\
Name & PRAVEEN KUMAR R\\
Grade & \\
Date & 19-02-2018\\
\end{tabular}
\end{center}


\section{Objective}
\label{sec-1}
\begin{enumerate}
\item Translate expressions to C.
\item Declare variables of data types appropriate for the calculation.
\item Order the update of variables using a sequence of assignments.
\item Use alternative and conditional statements.
\item Specify, define, and call simple functions.
\end{enumerate}

\section{Home work}
\label{sec-2}
\section{Area and Perimeter of circle}
\label{sec-3}

\subsection*{Problem Description}
\label{sec-3-1}
Write a program to calculate the area and the perimeter of a
circle. Read the radius from the user and print the outputs on the
display.

\subsection*{Specification}
\label{sec-3-2}
 \texttt{area(r)} 
  input: r of tynope float
  output: a value of type float calculated as pi*r*r.
\texttt{perimeter(r)}
  input: r of type float
  output: a value of type float calculated as 2* pi*r

\subsection*{Program Design}
\label{sec-3-3}
no   The program consists of the following.
\begin{enumerate}
\item \texttt{area(r)}: to compute the area of the circle with radius r.
\item \texttt{perimeter(r)}: to compute the perimeter of the circle with radius r.
\item \texttt{main()}:reads the radius and store it in r,
calls  the functions \texttt{area(r)}, \texttt{perimeter(r)} and prints the result.
\end{enumerate}

\subsection*{Program}
\label{sec-3-4}
\begin{verbatim}
#include<stdio.h>
/*
  input: r of type float
  output: a value of type float calculated as pi*r*r.

*/
float area(float r)
{
  return 3.14*r*r;
}
/*
  input: r of type float
  output: a value of type float calculated as 2* pi*r
*/
float perimeter(float r)
{
  return 2*3.14*r;
}
int main()
{
  float r,ar,p;
  scanf("%f",&r);
  ar=area(r);
  p=perimeter(r);
  printf("radius=%4.2f\t area=%4.2f\t perimeter=%4.2f\t",r,ar,p);
  return 0;
}
\end{verbatim}

\begin{center}
\begin{tabular}{lll}
radius=40.00 & area=5024.00 & perimeter=251.20\\
\end{tabular}
\end{center}

\section{Leap Year}
\label{sec-4}

\subsection*{Program Description}
\label{sec-4-1}
Write a Boolean function \texttt{is\_leap()} for testing whether a year
is leap year or not. Test the function from \texttt{main()}.

\subsection*{Specification}
\label{sec-4-2}
input: year, an variable of type int
output: FLASE, if the given year is not a leap year
        TRUE, if the given year is a leap year

\subsection*{Program Design}
\label{sec-4-3}
The program consists of the following.
\begin{enumerate}
\item \texttt{is\_leap(year)}: returns \texttt{true} if the year is leap and returns \texttt{false} if the year is not leap.
\item \texttt{main()}: inputs a year, tests the function \texttt{is\_leap(year)} and prints 1 if the year is leap and prints 0 if the year is not leap.
\end{enumerate}

\subsection*{Algorithm}
\label{sec-4-4}
is$_{\text{leap}}$(int year)
     if (year\%4==0 and year\%100!=0 or year\%400==0)
          return true
     else
          return false

\subsection*{Program}
\label{sec-4-5}
\begin{verbatim}
#include<stdio.h>
#include<stdbool.h>
bool is_leap(int year)
{
  if (year%4==0 && year%100!=0 ||year%400==0)
    return true;
  else
    return false;
}
int main()
{
  int year=1999,leap;
  // scanf("%d",&year);
  leap=is_leap(year);
  printf("%s",leap?"leap year":"not leap year");
}
\end{verbatim}
\subsection*{test input}
\label{sec-4-6}
\begin{verbatim}
1999
\end{verbatim}
\subsection*{output}
\label{sec-4-7}
\begin{verbatim}
not leap year
\end{verbatim}





\section{Roots of Quadratic equation}
\label{sec-5}

\subsection*{Program Description}
\label{sec-5-1}
Read the coefficients \texttt{a}, \texttt{b}, and \texttt{c} of a quadratic
equation. Calculate the discriminant.  Define a function \texttt{sign()}
that returns -1 or 0 or 1 for a negative number, zero or a positive
number, respectively. Use it to test the discriminant. If the
discriminant is non-negative, find the roots of the equation, and
print them. Avoid duplicate calculations whereever possible.

\subsection*{Specification}
\label{sec-5-2}
Function \texttt{sign(n)} takes the parameter n in float and returns one
of the integers -1,0 or 1.  \texttt{int sign(float n)}

\subsection*{Program Design}
\label{sec-5-3}
The program consists of the following.
\begin{enumerate}
\item \texttt{sign(n)}:returns 1 if n is positive, 0 if n is 0 and -1 if n
is negative.
\item \texttt{main()}:inputs a, b and c of a quadratic equation, then finds
the determinant \texttt{det}, tests the function \texttt{sign(n)} by passing
det as argument.  If \texttt{sign(n)} returns 1 or 0, finds the roots
and prints them. If it retuns -1, prints \textbf{imaginary roots}.
\end{enumerate}

\subsection*{Program}
\label{sec-5-4}
\begin{verbatim}
#include<stdio.h>
#include<math.h>
int sign(float n)
{
  if (n>0)
    return 1;
  else if(n==0)
    return 0;
  else
    return -1;
}
int main()
{
  int a,b,c,m;
  double det,r1,r2;
  // scanf("%d%d%d",&a,&b,&c);
  a=3;b=-6;c=-9;
  det=(b*b)-(4*a*c);
  m=sign(det);
  if (m==1)
    {
      r1=(-b+sqrt(det))/(2*a);
      r2=(-b-sqrt(det))/(2*a);
      printf("%lf %lf",r1,r2);
    }
  else if(m==0)
    {
      r1=-b/(2*a);
      printf("%lf",r1);
    }
  else
    printf("imaginary roots");
}
\end{verbatim}
\subsection*{Test}
\label{sec-5-5}
\subsubsection*{Input}
\label{sec-5-5-1}
1 -8 -7
\subsubsection*{Output}
\label{sec-5-5-2}
8.795832 -0.79583
\section{Distance between 2 points}
\label{sec-6}

\subsection*{Program Description}
\label{sec-6-1}
Write a program to compute the distance between two points. To read a point, the program should read 2 numbers from the user for the \texttt{x} and \texttt{y}
coordinates.  Hence your program should read numbers for the two points. Print the output on the stdout. Implement a function \texttt{distance(x1, y1, x2, y2)}
that takes two points \texttt{(x1, y1)} and \texttt{(x2, y2)} as 4 parameters and returns the distance between the two points. Avoid duplicate calculations
whereever possible.

\subsection*{Specification}
\label{sec-6-2}
Function \texttt{distance(x1,y1,x2,y2)} takes 4 parameters, the coordinates of 2 points in int and returns the distance between them using the formula
((x1-x2)$^{\text{2}}$ + (y1-y2)$^{\text{2}}$)$^{\text{1}}$/2 
\texttt{float distance(int x1,int y1,int x2,int y2)}

\subsection*{Program Design}
\label{sec-6-3}
The program consists of the following.
\begin{enumerate}
\item \texttt{distance(x1,y1,x2,y2)}: returns the distance between the points (x1,y1) and (x2,y2).
\item \texttt{main()}:inputs the two points, tests the function \texttt{distance(x1,y1,x2,y2)} and prints the result.
\end{enumerate}

\subsection*{Program}
\label{sec-6-4}

\begin{verbatim}
#include<stdio.h>
#include<math.h>
float distance(int x1,int y1 ,int x2,int y2)
{
   float dis;
   dis=sqrt((x1-x2)*(x1-x2)+(y1-y2)*(y1-y2));
   return dis;
}
int main()
{
   int x1,x2,y1,y2;
   float dis;
   scanf("%d %d %d %d",&x1,&y1,&x2,&y2);
   dis=distance(x1,y1,x2,y2);
   printf("%f",dis);
}
\end{verbatim}
\subsection*{Test}
\label{sec-6-5}
\subsubsection*{Input}
\label{sec-6-5-1}
4 5 -3 3
\subsubsection*{Output}
\label{sec-6-5-2}
7.280110
\section{Swap two variables.}
\label{sec-7}

\subsection*{Program Description}
\label{sec-7-1}
Initialize two variables with values read from the user and exchange (swap) their contents. Print them before and after the swap.

\subsection*{Program Design}
\label{sec-7-2}
The program consists of the following.
\begin{enumerate}
\item \texttt{main()}:inputs two numbers, a and b,prints them before and after swapping.
\end{enumerate}

\subsection*{Program}
\label{sec-7-3}
\begin{verbatim}
#include <stdio.h>

int main ()
{
  int a;
  int b;

  scanf ("%d%d", &a, &b);
  printf ("a = %d, b = %d\n", a, b);
  int t = a;
  a = b;
  b = t;
  printf ("a = %d, b = %d\n", a, b);  
  return 0;
}
\end{verbatim}

\begin{center}
\begin{tabular}{ll}
a = 1 & b = -3\\
a = -3 & b = 1\\
\end{tabular}
\end{center}

\section{Swap using function}
\label{sec-8}
\subsection*{Description}
\label{sec-8-1}
Define a function \texttt{swap()} to exchange the contents of the two variables, and check whether the function works as intended. If it does not work,
what is the reason?

\subsection*{Specification}
\label{sec-8-2}
Function \texttt{swap(a,b)} takes two numbers as parameters and returns the numbers after swapping them.

\subsection*{Function}
\label{sec-8-3}
int swap(int a,int b)
\{
   int t;
   t=a;
   a=b;
   b=t;
   return (a,b);
)

\subsection*{Reason}
\label{sec-8-4}
This function does not work because a function in C programming language can return only one value. It cannot return multiple values.

\section{Circulate numbers}
\label{sec-9}

\subsection*{Program Description}
\label{sec-9-1}
Read four numbers \texttt{a, b, c, d} from stdin. Circulate them so that a gets the value of b, and so on: \texttt{a <- b <- c <- d <- a}

\subsection*{Program Design}
\label{sec-9-2}
The program consists of the following.
\begin{enumerate}
\item \texttt{main()}: inputs 4 numbers, circulates them in the way \texttt{a <- b <- c <- d <- a} and prints the numbers after circulation.
\end{enumerate}

\subsection*{Algorithm}
\label{sec-9-3}
if there are n numbers from a$_{\text{1}}$ to a$_{\text{n}}$,
  a1=t;
  a1=a2;
  a2=a3;
  .
  .
  .
  a$_{\text{(n-1)}}$=a$_{\text{n}}$;
  a$_{\text{n}}$=t;

\subsection*{Program}
\label{sec-9-4}
\begin{verbatim}
#include <stdio.h>

int main ()
{
  int a, b, c, d,t;

  scanf ("%d%d%d%d", &a, &b, &c, &d);
  printf ("a = %d, b = %d, c = %d, d = %d\n", a, b, c, d);
  t = a;
  a = b;
  b = c;
  c = d;
  d = t;
  printf ("a = %d, b = %d, c = %d, d = %d\n", a, b, c, d);
  return 0;
}
\end{verbatim}

\begin{center}
\begin{tabular}{llll}
a = 1 & b = -3 & c = 2 & d = 6\\
a = -3 & b = 2 & c = 6 & d = 1\\
\end{tabular}
\end{center}

\section{Rearrange three numbers}
\label{sec-10}

\subsection*{Program Description}
\label{sec-10-1}
Read three numbers \texttt{a, b, c} from stdin. Write a program to rearrange them so that \texttt{a \$\textbackslash{}le\$ b \$\textbackslash{}le\$ c}.

\subsection*{Specification}
\label{sec-10-2}
Function \texttt{min2(a,b)} takes two integer parameters and returns the minimum of the two.
Function \texttt{min3(a,b,c)} takes three integer parameters and returns the minimum of the three.
\texttt{int min2(int a,int b)}
\texttt{int min3(int a,int b,int c)}

\subsection*{Program Design}
\label{sec-10-3}
The program consists of the following.
\begin{enumerate}
\item \texttt{min2(a,b)}:returns minimum of two numbers.
\item \texttt{min3(a,b,c)}:returns minimum of three numbers.
\item \texttt{main()}:inputs three numbers,tests \texttt{min2(a,b)} and \texttt{min3(a,b)} and rearranges the three numbers a,b,c such that \texttt{a \$\textbackslash{}le\$ b \$\textbackslash{}le\$ c}.
\end{enumerate}

\subsection*{Program}
\label{sec-10-4}
\begin{verbatim}
#include<stdio.h>
int min2(int a,int b)
{
   if(a<=b)
       return a;
   else
       return b;
}
int min3(int a, int b, int c)
{
   int temp;
   temp=min2(a,b);
   return min2(temp,c);
}
int main()
{
   int a,b,c,s,t,u;
   scanf("%d%d%d",&a,&b,&c);
   printf("a=%d, b=%d, c=%d\n",a,b,c);
   t=min3(a,b,c);
   if (t==a)
   {
       s=min2(b,c);
       if (s==b)
	  u=c;
       else
	  u=b;
   }
   else if (t==b)
	{
	    s=min2(a,c);
	    if(s==a)
	       u=c;
	    else u=a;
	} 
	else
	{
	    s=min2(a,b);
	    if(s==a)
	       u=b;
	    else
	       u=a;
	}
   a=t;
   b=s;
   c=u;
   printf("a=%d, b=%d, c=%d",a,b,c);
}
\end{verbatim}

\begin{center}
\begin{tabular}{lll}
a = 1 & b = -3 & c = 2\\
a = -3 & b = 1 & c = 2\\
\end{tabular}
\end{center}

\section{Rearrange numbers in an array}
\label{sec-11}

\subsection*{Program Description}
\label{sec-11-1}
Fill an array of 3 numbers with numbers read from stdin. Write a program to rearrange them so that \texttt{a[0] \$\textbackslash{}le\$ a[1] \$\textbackslash{}le\$ a[2]}

\subsection*{Specification}
\label{sec-11-2}
Function \texttt{min2(a,b)} takes two integer parameters and returns the minimum of the two.
Function \texttt{min3(a,b,c)} takes three integer parameters and returns the minimum of the three.
\texttt{int min2(int a,int b)}
\texttt{int min3(int a,int b,int c)}

\subsection*{Program Design}
\label{sec-11-3}
The program consists of the following.
\begin{enumerate}
\item \texttt{min2(a,b)}:returns minimum of two numbers.
\item \texttt{min3(a,b,c)}:returns minimum of three numbers.
\item \texttt{main()}:inputs three numbers and stores them in an array, tests \texttt{min2(a,b)} and \texttt{min3(a,b)} and rearranges the three numbers in the array 
such that \texttt{a[0]} $\le$ \texttt{a[1]} $\le$ \texttt{a[2]}.
\end{enumerate}

\subsection*{Program}
\label{sec-11-4}
\begin{verbatim}
#include<stdio.h>
int min2(int a,int b)
{
   if(a<=b)
      return a;
   else
      return b;
}
int min3(int a, int b, int c)
{
   int temp;
   temp=min2(a,b);
   return min2(temp,c);
}
int main()
{
   int a[5],s,t,u,i;
   for (i=0;i<3;i++)
   {
      scanf("%d",&a[i]);
      printf("a[%d]=%d",i,a[i]);
      if (i==2) 
	  printf("\n");
      else
	  printf(", ");
   }  
   t=min3(a[0],a[1],a[2]);
   if (t==a[0])
   {
       s=min2(a[1],a[2]);
       if (s==a[1])
	  u=a[2];
       else u=a[1];
   }
   else if (t==a[1])
	{
	   s=min2(a[0],a[2]);
	   if(s==a[0])
	      u=a[2];
	   else u=a[0];
	}
	else
	{ 
	   s=min2(a[0],a[1]);
	   if(s==a[0])
	      u=a[1];
	   else u=a[0];
	}
   a[0]=t;
   a[1]=s;
   a[2]=u;
   printf("a[0]=%d, a[1]=%d, a[2]=%d",a[0],a[1],a[2]);
 }
\end{verbatim}

\begin{center}
\begin{tabular}{lll}
a\footnotemark = 1 & a\footnotemark = -3 & a\footnotemark = 2\\
a\footnotemark[1]{} = -3 & a\footnotemark[2]{} = 1 & a\footnotemark[3]{} = 2\\
\end{tabular}
\end{center}\footnotetext[1]{DEFINITION NOT FOUND.}\footnotetext[2]{DEFINITION NOT FOUND.}\footnotetext[3]{DEFINITION NOT FOUND.}
% Emacs 24.5.1 (Org mode 8.2.10)
\end{document}
