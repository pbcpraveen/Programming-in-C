% Created 2018-04-15 Sun 01:28
\documentclass[11pt]{article}
\usepackage[utf8]{inputenc}
\usepackage[T1]{fontenc}
\usepackage{fixltx2e}
\usepackage{graphicx}
\usepackage{longtable}
\usepackage{float}
\usepackage{wrapfig}
\usepackage{rotating}
\usepackage[normalem]{ulem}
\usepackage{amsmath}
\usepackage{textcomp}
\usepackage{marvosym}
\usepackage{wasysym}
\usepackage{amssymb}
\usepackage{hyperref}
\tolerance=1000
\usepackage{palatino}
\usepackage[top=1in, bottom=1.25in, left=1.25in, right=1.25in]{geometry}
\usepackage{setspace}
\setcounter{secnumdepth}{1}
\author{PRAVEEN KUMAR R}
\date{20-02-2018}
\title{Exercise 3: Conditional and Alternative statements}
\hypersetup{
  pdfkeywords={},
  pdfsubject={},
  pdfcreator={Emacs 24.5.1 (Org mode 8.2.10)}}
\begin{document}

\maketitle
%\linespread{1.2}
\begin{center}
\begin{tabular}{lr}
Assignment & 3\\
Reg No & 312217104114\\
Name & PRAVEEN KUMAR R\\
Grade & \\
Date & 20-02-2018\\
\end{tabular}
\end{center}


\section{24 hour format}
\label{sec-1}
Write a program to get a time in 24 hour format and convert it to a 12 hour format
\subsection*{Program Design}
\label{sec-1-1}
The program consists of \texttt{main()}, which gets the input of time from \texttt{stdin}, 
converts it to 12 hour format and prints the result on \texttt{stdout}.
\subsection*{Algorithm}
\label{sec-1-2}
def func(h,m,s):
  if h<=12:
    if h==12:
      print("\%d:\%d:\%d pm\n",h,m,s)
    else
      print("\%d:\%d:\%d am\n",h,m,s)
  else
    if h==24:
      print("\%d:\%d:\%d am\n",h-24,m,s)
    else
      print("\%d:\%d:\%d pm\n",h-12,m,s)

\subsection*{Source Code:}
\label{sec-1-3}

\begin{verbatim}
#include<stdio.h>
int main()
{
  int h,m,s;
  scanf("%d%d%d",&h,&m,&s);
  printf("%d:%d:%d %s",h<13?h:h-12,m,s,h<13?"am":"pm");  
}
\end{verbatim}

\subsection*{Test Input:}
\label{sec-1-4}
\begin{verbatim}
23 15 45
\end{verbatim}
\subsection*{Output:}
\label{sec-1-5}
\begin{verbatim}
11:15:45 pm
\end{verbatim}

\section{Time Comparison}
\label{sec-2}
Write a function to accept 2 time in hours minutes and seconds and compare which time 
is earlier.
\subsection*{Program Design}
\label{sec-2-1}
The program consists of \texttt{main()}, which gets the input from \texttt{stdin}, compares the
times and prints the result on \texttt{stdout}.
\subsection*{Algorithm}
\label{sec-2-2}

\begin{verbatim}
def func(h1,m1,s1,h2,m2,s2):
  if h1>h2:
    print("t1 is earlier")
  elif h1<h2:
    print("t2 is earlier")
  else
    if m1>m2:
      print("t1 is earlier")
    elif m1<m2:
      print("t2 is earlier")
    else:
      if s1>s2:
        print("t1 is earlier")
      elif s1<s2:
        print("t2 is earlier")
      else:
        print("Both are same")
\end{verbatim}
\subsection*{Source Code:}
\label{sec-2-3}
\begin{verbatim}
#include<stdio.h>
int compare(int h1,int m1,int s1,int h2,int m2,int s2)
{
  if(h1<h2)
    {
      return -1;
    }
  else if(h1>h2)
    {
      return 1;
    }
  else
    {
      if(m1>m2)
	{
	  return 1;
	}
      else if(m1<m2)
	{
	  return -1;
	}
      else
	{
	  if(s1>s2)
	    {
	      return 1;
	    }
	  else if(s1<s2)
	    {
	      return -1;
	    }
	  else
	    {
	      return 0;
	    }
	}     
    }
}
int main()
{
  int h1,m1,s1,h2,m2,s2;
  scanf("%d%d%d",&h1,&m1,&s1);
  scanf("%d%d%d",&h2,&m2,&s2);
  printf("%d %d %d\t%d %d %d\n",h1,m1,s1,h2,m2,s2);
  printf("%d ",compare(h1,m1,s1,h2,m2,s2);
  return 0;
}
\end{verbatim}

\subsection*{Test Input:}
\label{sec-2-4}
\begin{verbatim}
14 6 34  14 6 33
\end{verbatim}
\subsection*{Output:}
\label{sec-2-5}
\begin{verbatim}
-1          
\end{verbatim}

\section{Time difference}
\label{sec-3}
Write a program to calculate the time difference between the two 
time the user enters and print it
\subsection*{Specification}
\label{sec-3-1}
A function \texttt{sign()}, which takes an integer as the input and returns
it's sign to the calling function.
\subsection*{Prototype}
\label{sec-3-2}
\begin{verbatim}
int sign(int a);
\end{verbatim}
\subsection*{Program Design}
\label{sec-3-3}
The program consists a function \texttt{sign(int a)}, which returns the 
sign of the integer, and \texttt{main()}, which gets the input from \texttt{stdin},
calls the function and prints the rsult accordingly on \texttt{stdout}.
\subsection*{Algorithm}
\label{sec-3-4}
\begin{verbatim}
def sign(a):
  if a>=0:
    return 1
  else
    return -1
\end{verbatim}
\subsection*{Sorce Code}
\label{sec-3-5}

\begin{verbatim}
#include<stdio.h>

int sign(int a){
  if(a>=0){
    return 1;
  }
  else{
    return -1;
  }
}

int main(){
  int a,b,c,d,e,f,g,h,i;
  scanf("%d%d%d",&a,&b,&c);
  scanf("%d%d%d",&d,&e,&f);
  g=sign(a-d);
  h=sign(b-e);
  i=sign(c-f);
  if(g>0){
    if(h>0 && i>0){
      printf("%d:%d:%d\n",a-d,b-e,c-f);
    }
    else if(h>0 && i<0){
      printf("%d:%d:%d\n",a-d,b-e,f-c);
    }
    else if(h<0 && i>0){
      printf("%d:%d:%d\n",a-d,e-b,c-f);
    }
    else{
      printf("%d:%d:%d\n",a-d,e-b,f-c);
    }  
  }
  else{
    if(h>0 && i>0){
      printf("%d:%d:%d\n",d-a,b-e,c-f);
    }
    else if(h>0 && i<0){
      printf("%d:%d:%d\n",d-a,b-e,f-c);
    }
    else if(h<0 && i>0){
      printf("%d:%d:%d\n",d-a,e-b,c-f);
    }
    else{
      printf("%d:%d:%d\n",d-a,e-b,f-c);
    }  
  }
}
\end{verbatim}

\subsection*{Test Input}
\label{sec-3-6}
\begin{verbatim}
14 6 34        14 6 33
\end{verbatim}
\subsection*{Output}
\label{sec-3-7}
\begin{verbatim}
0 0 1                 
\end{verbatim}

\section{Smallest and largest of 4 numbers}
\label{sec-4}
Write a program to find the smallest and largest number out of the 4 numbers entered 
from the standard input
\subsection*{Specification}
\label{sec-4-1}
2 functions \texttt{min2()} and \texttt{max2()}, which take 2 integers as the input and returns
the minimum and maximum of the two to the calling function respectively.
\subsection*{Prototype}
\label{sec-4-2}
\begin{verbatim}
int min2(int a, int b);
int max2(int a, int b);
\end{verbatim}
\subsection*{Program Design}
\label{sec-4-3}
The program consists of 2 functions \texttt{min2(int a, int b)} and \texttt{max2(int a, int b)}
which returns the minimum and maximum of the 2 numbers, and \texttt{main()}, which
gets the input from \texttt{stdin}, calls the functions, and prints the result on \texttt{stdout}.
\subsection*{Algorithm}
\label{sec-4-4}
\begin{verbatim}
def min2(a,b):
  if a>b:
    return b
  else:
    return a
def max2(a,b):
  if a<b:
    return b
  else:
    return a
\end{verbatim}
\subsection*{Source Code}
\label{sec-4-5}

\begin{verbatim}
#include<stdio.h>
int min2(int a, int b){
  if(a>b){
    return b;
  }
  else{
    return a;
  }
}

int max2(int a, int b){
  if(a<b){
    return b;
  }
  else{
    return a;
  }
}

int main(){
  int a,b,c,d,m,n;
  scanf("%d%d%d%d",&a, &b, &c, &d);
  m=min2(a,b);
  m=min2(m,c);
  m=min2(m,d);
  n=max2(a,b);
  n=max2(n,c);
  n=max2(n,d);
  printf("%d,%d\n",m,n);
}
\end{verbatim}

\subsection*{Test Input}
\label{sec-4-6}
\begin{verbatim}
-98 56 928 -999
\end{verbatim}
\subsection*{Output}
\label{sec-4-7}
\begin{verbatim}
-999 928
\end{verbatim}

\section{Grades}
\label{sec-5}
Write a function \texttt{grades()} to translate the marks of a student in various subjects 
into letter grades and print the grades on the output.
\begin{center}
\begin{tabular}{rrl}
Mark range & Grade points & Leter grade\\
91-100 & 10 & S\\
81-90 & 9 & A\\
71-80 & 8 & B\\
61-70 & 7 & C\\
57-60 & 6 & D\\
51-56 & 5 & E\\
<50 & 0 & U\\
\end{tabular}
\end{center}
\subsection*{Specification}
\label{sec-5-1}
A function \texttt{grade()}, which gets the mark as the input and returns a grade as 
character to the calling function.
\subsection*{Prototype}
\label{sec-5-2}
\begin{verbatim}
char grade(int x);
\end{verbatim}
\subsection*{Program Design}
\label{sec-5-3}
The program consists of a function \texttt{grade(int x)}, which returns a grade as a 
character based on the mark, and \texttt{main()}, which gets the input from \texttt{stdin},
calls the function and prints the result on \texttt{stdout}.
\subsection*{Algorithm}
\label{sec-5-4}
\begin{verbatim}
def grade(x):
  if x>90:
    return 's'
  elif x>80:
    return 'a'
  elif x>70:
    return 'b'
  elif x>60:
    return 'c'
  elif x>56:
    return 'd'
  elif x>50:
    return 'e'
  else:
    return 'u'
\end{verbatim}
\subsection*{Source Code}
\label{sec-5-5}

\begin{verbatim}
#include<stdio.h>
char grade(int x){
  if(x>90){
    return 'S';
  }
  else if(x>80){
    return 'A';
  }
  else if(x>70){
    return 'B';
  }
  else if(x>60){
    return 'C';
  }
  else if(x>56){
    return 'D';
  }
  else if(x>50){
    return 'E';
  }
  else{
    return 'U';
  }
}  
int main(){
  int a[20],n;
  char g;
  scanf("%d",&n);
  for(int i=0;i<n;i++){
    scanf("%d",&a[i]);
  }
  for(int i=0;i<n;i++){
    g=grade(a[i]);
    printf("%c\n",g);
  }
}
\end{verbatim}

\subsection*{Test Input}
\label{sec-5-6}
\begin{verbatim}
8
94 56 33 78 81 99 100 66
\end{verbatim}
\subsection*{Output}
\label{sec-5-7}
\begin{verbatim}
S 
E 
U 
B 
A 
S
S 
C 
\end{verbatim}

\section{Tariff Calculator}
\label{sec-6}
Write a function \texttt{eb()} to find out the domestic eb bill based on the given slab rates
\begin{enumerate}
\item Consumption upto 100 units: free.
\item Consumption above 100 units and upto 200 units: Rs 1.50 per unit.
\item Consumption above 200 units and upto 500 units: Rs 2.00 per unit 
for 101-200 units and Rs 3.00 per unit for 201-500 units.
\item Consumption above 500 units: Rs 3.50 per unit for 101-200 units, 
Rs 4.60 per unit for 201-500 units, and Rs 6.60 beyond 500 units.
\end{enumerate}
\subsection*{Specification}
\label{sec-6-1}
A function \texttt{eb()}, which takes the number of units as the input and returns the cost
based on the conditions to the calling function.
\subsection*{Prototype}
\label{sec-6-2}
\begin{verbatim}
float eb(int unit);
\end{verbatim}
\subsection*{Program Design}
\label{sec-6-3}
The program consists of a function \texttt{eb(int unit)}, which returns the net cost, and \texttt{main()},
which gets the input from \texttt{stdin}, calls the function and prints the result on \texttt{stdout}.
\subsection*{Algorithm}
\label{sec-6-4}
\begin{verbatim}
def eb(u):
  if u<=100:
    return 0
  elif u>100 and u<=200:
    return 1.5*u
  elif u>200 and u<=500:
    return (u-200)*3.0+(u-100)*2.0
  else:
    return (u-500)*6.6+(u-200)*4.6+(u-100)*3.5
\end{verbatim}
\subsection*{Source Code}
\label{sec-6-5}

\begin{verbatim}
#include<stdio.h>
float eb(int unit){
  if(unit<=100){
    return 0.0;
  }
  else if((unit>100)&&(unit<=200)){
    return 1.5*unit;
  }
  else if((unit>200)&&(unit<=500)){
    return(unit-200)*3.0+100*2.0;
  }
  else{
    return (unit-500)*6.6+300*4.6+100*3.5;
  }
}
int main(){
  int unit;
  float cost;
  scanf("%d",&unit);
  cost=eb(unit);
  printf("%.4f\n",cost);
}
\end{verbatim}

\subsection*{Test Input}
\label{sec-6-6}
\begin{verbatim}
345
\end{verbatim}
\subsection*{Output}
\label{sec-6-7}
\begin{verbatim}
635.0
\end{verbatim}

\section{Income Tax}
\label{sec-7}
Write a function \texttt{tax()} to calculate the income tax based on the age and the income 
of the person
\begin{enumerate}
\item Income Tax Slab for Individual Tax Payers (Less Than 60 Years Old)
\end{enumerate}
\begin{center}
\begin{tabular}{ll}
Income Slab & Tax Rate\\
Up to Rs.2,50,000 & No tax\\
Rs.2,50,000 - Rs.5,00,000 & 5\%\\
Rs.5,00,000 - Rs.10,00,000 & 20\%\\
Rs.10,00,000 and beyond & 30\%\\
\end{tabular}
\end{center}
\begin{enumerate}
\item Income Tax Slab for Senior Citizens (60 Years Old Or more but Less than 80 Years Old)
\end{enumerate}
\begin{center}
\begin{tabular}{ll}
Income Slab & Tax Rate\\
Up to Rs.3,00,000 & No tax\\
Rs.3,00,000 - Rs.5,00,000 & 5\%\\
Rs.5,00,000 - Rs.10,00,000 & 20\%\\
Rs.10,00,000 and beyond & 30\%\\
\end{tabular}
\end{center}
\begin{enumerate}
\item Income Tax Slab for Senior Citizens (More than 80 years old)
\end{enumerate}
\begin{center}
\begin{tabular}{ll}
Income Slab & Tax Rate\\
Up to Rs.2,50,000 & No tax\\
Rs.2,50,000 - Rs.5,00,000 & No tax\\
Rs.5,00,000 - Rs.10,00,000 & 20\%\\
Rs.10,00,000 and beyond & 30\%\\
\end{tabular}
\end{center}
Modify your function to take the age and the income as the parameters and calculate the tax.
\subsection*{Specification}
\label{sec-7-1}
A function \texttt{tax()}, which gets the age and income as the inputs, checks the conditions
and returns the value of tax to the calling function
\subsection*{Prototype}
\label{sec-7-2}
\begin{verbatim}
float tax(int age, int income);
\end{verbatim}
\subsection*{Program Design}
\label{sec-7-3}
The program consists of a function \texttt{tax(int age, int income)}, which returns the value
of tax based on conditions, and \texttt{main()}, which gets the input from \texttt{stdin}, calls
the function and prints the result on \texttt{stdout}.
\subsection*{Algorithm}
\label{sec-7-4}
\begin{verbatim}
def tax(age,income):
  if(age<60):
      if income<250000:
        return 0.0
      elif income>=250000 and income<500000:
        return (5.0/100)*income
      elif income>=500000 andincome<1000000:
        return (20.0/100)*income
      else:
        return (30.0/100)*income
    else ifage>=60 and age<80:
      if income<300000;
        return 0.0
      elif income>=300000 and income<500000:
        return (5.0/100)*income
      elif income>=500000 and income<1000000:
        return (20.0/100)*income
      else:
        return (30.0/100)*income
    else:
      if income<500000:
        return 0.0
      elif income>=500000 and income<1000000:
        return (20.0/100)*income
      else:
        return(30.0/100)*income
\end{verbatim}
\subsection*{Source Code}
\label{sec-7-5}

\begin{verbatim}
#include<stdio.h>
float tax(int age, int income)
{
  if(age<60)
    {
      if(income<250000)
      {
	return 0.0;
      }
      else if((income>=250000)&&(income<500000))
	{
	  return (5.0/100)*income;
	}
      else if((income>=500000)&&(income<1000000))
	{
	  return (20.0/100)*income;
	}
      else
	{
	  return (30.0/100)*income;
	}
    }
  else if((age>=60)&&(age<80))
    {
      if(income<300000)
	{
	  return 0.0;
	}
      else if((income>=300000)&&(income<500000))
	{
	  return (5.0/100)*income;
	}
      else if((income>=500000)&&(income<1000000))
	{
	  return (20.0/100)*income;
	}
      else
	{
	  return (30.0/100)*income;
	}
    }
  else
    {
      if(income<500000)
	{
	  return 0.0;
	}
      else if((income>=500000)&&(income<1000000))
	{
	  return (20.0/100)*income;
	} 
      else
	{
	  return(30.0/100)*income;
	}
    }
}
int main()
{
  int age,income;
  float t;
  scanf("%d%d",&age,&income);
  t=tax(age,income);
  printf("%f\n",t);
}
\end{verbatim}

\subsection*{Test Input}
\label{sec-7-6}
\begin{verbatim}
45 344000
\end{verbatim}
\subsection*{Output}
\label{sec-7-7}
\begin{verbatim}
17200.0        
\end{verbatim}







\section{Inversion}
\label{sec-8}
In a sequence of integers \texttt{a0, a1, a2, a3}, any pair of integers \texttt{(ai, aj)} 
is said to be an \emph{inversion} if \texttt{ai > aj} for \texttt{i < j}. Write a program to 
correct/order all the inversions in the        sequence. 
\subsection*{Specification}
\label{sec-8-1}
A function \texttt{inversion()}, which takes an array and it's length as input,
counts the number of inversions to be performed and returns the result
to the calling function.
\subsection*{Prototype}
\label{sec-8-2}
\begin{verbatim}
int inversion(int a[], int n);
\end{verbatim}
\subsection*{Program Design}
\label{sec-8-3}
The program consists of a function \texttt{inversion(int a[], int n)}, which
counts the number of inversions to be done, and \texttt{main()}, which gets 
the input from \texttt{stdin}, calls the function, and prints the result
on \texttt{stdout}.
\subsection*{Algorithm}
\label{sec-8-4}
\begin{verbatim}
def inversion(a,n):
  c=0
  for i in range(n):
    for j in range(i+1,n):
      if a[i]>a[j]:
        c+=1
  return c
\end{verbatim}
\subsection*{Source Code}
\label{sec-8-5}
\begin{verbatim}
#include<stdio.h>
int inversion(int a[], int n)
{
  int c=0;
  for(int i=0;i<n;i++)
    {
      for(int j=i+1;j<n;j++)
	{
	  if(a[i]>a[j]){
	    c++;
	  }
	}
    }
  return c;
}
int main()
{
  int a[20],n;
  scanf("%d",&n);
  for(int i=0;i<n;i++)
    {
      scanf("%d",&a[i]);
    }
  int c=inversion(a,n);
  printf("%d",c);
}
\end{verbatim}
\subsection*{Test Input}
\label{sec-8-6}
\begin{verbatim}
10
34 31 43 13 434 -9238 1334 1244 366
\end{verbatim}
\subsection*{Output}
\label{sec-8-7}
\begin{verbatim}
21
\end{verbatim}
% Emacs 24.5.1 (Org mode 8.2.10)
\end{document}
