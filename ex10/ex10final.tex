% Created 2018-04-11 Wed 12:45
\documentclass[11pt]{article}
\usepackage[utf8]{inputenc}
\usepackage[T1]{fontenc}
\usepackage{fixltx2e}
\usepackage{graphicx}
\usepackage{longtable}
\usepackage{float}
\usepackage{wrapfig}
\usepackage{rotating}
\usepackage[normalem]{ulem}
\usepackage{amsmath}
\usepackage{textcomp}
\usepackage{marvosym}
\usepackage{wasysym}
\usepackage{amssymb}
\usepackage{hyperref}
\tolerance=1000
\usepackage{palatino}
\usepackage[top=1in,bottom=1.25in,left=1.25in,right=1.25in]{geometry}
\usepackage{setspace}
\author{PRAVEEN KUMAR R}
\date{10/04/18}
\title{Exercise 10: Structures}
\hypersetup{
  pdfkeywords={},
  pdfsubject={},
  pdfcreator={Emacs 24.5.1 (Org mode 8.2.10)}}
\begin{document}

\maketitle
\begin{export}
\linespread{1.2}
\end{export}
\begin{center}
\begin{tabular}{lr}
Assignment & 10\\
Reg No & 312217104114\\
Name & PRAVEEN KUMAR R\\
Grade & \\
Date & 19-02-2018\\
\end{tabular}
\end{center}


\section{Array of structures}
\label{sec-1}
\subsection{Problem description}
\label{sec-1-1}

\begin{enumerate}
\item Define a structure to represent students. It should store 3 ut marks for a subject and
the final internal mark for that subject.
\end{enumerate}
\begin{verbatim}
struct student {
  int rollnum;
  char name[100];
  int ut[4];
};
\end{verbatim}

Write functions to create a student structure and initialzie it. Write a function to print
a student struture.
\begin{enumerate}
\item Read the roll numbers, names and ut marks for 3 unit tests for 10 students from stdin.
\item Write a function to calculate the final internal mark for each students.
\item Modify the structure to store the ut marks and internal mark in 5 subjects. Write a
function to compute internal marks of students for 5 different subjects.
\end{enumerate}
\subsubsection{Subproblem 1}
\label{sec-1-1-1}
\begin{verbatim}
nilbody
\end{verbatim}

\begin{center}
\begin{tabular}{rll}
1 & praveen & 45 67 87\\
\end{tabular}
\end{center}

\subsubsection{Subproblem 2}
\label{sec-1-1-2}
\begin{verbatim}
  #include<stdio.h>
  #include<string.h>
  #include<ctype.h>
  #include<stdlib.h>
  #define MAXLINE 100
  struct student{
    int rollnum;
    char name[100];
    int ut[4];
    int avg; 
  };
  typedef struct student Student;
  int read(Student s[])
  {
    int i=0;
    char line[MAXLINE];
    for( i=0;fgets(line,MAXLINE,stdin);i++)
      {
	sscanf(line,"|%d| %s |%d|%d|%d|",
	       &s[i].rollnum,s[i].name,&s[i].ut[0],&s[i].ut[1],&s[i].ut[2]);

      }
    return i;
  }
 void cal_average(Student s[],int n)
{
   float f;
  for(int i=0;i<n;i++)
    {
       f=(s[i].ut[1]+s[i].ut[0]+s[i].ut[2])/15.0;
       s[i].avg= f-(int)f>0.5?(int)f+1:(int)f;
    }
}
  void print_all(Student s[],int n)
  {
    for(int i=0;i<n;i++)
	printf("%d\t%s\t%d %d %d  %d\n",
		s[i].rollnum,s[i].name,s[i].ut[0],s[i].ut[1],s[i].ut[2],s[i].avg);

  }

  int main()
  {
    Student s[100];
    int n=read(s);
    cal_average(s,n);
    print_all(s,n); 
    return 0;
  }
\end{verbatim}

\begin{center}
\begin{tabular}{rll}
1 & praveenkumar & 99 98 100  20\\
2 & ramkaushik & 97 100 99  20\\
3 & nitin & 99 98 100  20\\
4 & pratheeps & 90 89 100  19\\
5 & Nandini & 100 94 95  19\\
\end{tabular}
\end{center}

\subsubsection{Subproblem 3}
\label{sec-1-1-3}
\begin{verbatim}
#include<stdio.h>
#include<string.h>
#include<ctype.h>
#include<stdlib.h>
#define MAXLINE 100
struct student{
  int rollnum;
  char name[100];
  int ut[5];
    int avg; 
};
typedef struct student Student;
int read(Student s[])
{
  int i=0;
  char line[MAXLINE];
  for( i=0;fgets(line,MAXLINE,stdin);i++)
    {
      sscanf(line,"|%d| %s |%d|%d|%d|%d|%d|",
	     &s[i].rollnum,s[i].name,&s[i].ut[0],&s[i].ut[1],&s[i].ut[2],&s[i].ut[3],&s[i].ut[4]);

    }
  return i;
}
void cal_average(Student s[],int n)
{
  float f;
  for(int i=0;i<n;i++)
    {
      f=(s[i].ut[1]+s[i].ut[0]+s[i].ut[2]+s[i].ut[3]+s[i].ut[4])/25.0;
      s[i].avg= f-(int)f>0.5?(int)f+1:(int)f;
    }
}
void print_all(Student s[],int n)
{
  for(int i=0;i<n;i++)
	printf("%d\t%s\t%d %d %d %d %d   %d  \n",
	   s[i].rollnum,s[i].name,s[i].ut[0],s[i].ut[1],s[i].ut[2],s[i].ut[3],s[i].ut[4],s[i].avg);

}

int main()
{
  Student s[100];
  int n=read(s);
  cal_average(s,n);
  print_all(s,n); 
  return 0;
}
\end{verbatim}
\subsubsection{Sample input}
\label{sec-1-1-4}
The contents of the ex10$_{\text{1}}$.in file is
\begin{center}
\begin{tabular}{rlrrrrr}
1 & praveenkumar & 99 & 98 & 100 & 97 & 99\\
2 & ramkaushik & 97 & 100 & 99 & 87 & 94\\
3 & nitin & 99 & 98 & 100 & 84 & 95\\
4 & pratheeps & 90 & 89 & 100 & 88 & 90\\
5 & Nandini & 100 & 94 & 95 & 93 & 96\\
\end{tabular}
\end{center}


\subsubsection{Output}
\label{sec-1-1-5}

\begin{center}
\begin{tabular}{rll}
1 & praveenkumar & 99 98 100 97 99   20\\
2 & ramkaushik & 97 100 99 87 94   19\\
3 & nitin & 99 98 100 84 95   19\\
4 & pratheeps & 90 89 100 88 90   18\\
5 & Nandini & 100 94 95 93 96   19\\
\end{tabular}
\end{center}

\section{Number conversion}
\label{sec-2}
\subsection{problem description}
\label{sec-2-1}
Convert the given decimal number into binary, octal and hexadecimal numbers using user
defined functions.
\begin{verbatim}
#include<stdio.h>
#include<ctype.h>
#include<stdlib.h>
#include<string.h>
int k=0;
int binary(int n,char a[])
{

  if(n<=1)
    a[k++]=(char)(n+48);
  else 
    {
      int p=n%2;
      binary(n/2,a);
      a[k++]=(char)(p+48);
    }
  return k;
}
int octal(int n,char a[])
{
  if(n<=7)
    a[k++]=(char)(n+48);
  else 
    {
      int p=n%8;
      octal(n/8,a);
      a[k++]=(char)(p+48);
    }
  return k;
}
int hexadecimal(int n,char a[])
{
  if(n<=15)
    {
      if(n>=10)
	a[k++]='A'+(n-10);
      else
	a[k++]=(char)(n+48);
    }
  else 
    {
      int p=n%16;
      hexadecimal(n/16,a);
      if(p>=10)
	a[k++]='A'+(p-10);
      else 
	a[k++]=(char)(p+48);
    }
  return k;
}
int main()
{
  int n=45;
  char c1[100],c2[100],c3[100];
  int p1,p2,p3;
  while (scanf("%d",&n)!=EOF)
    {
      p1=binary(n,c1);
      c1[p1]='\0';
      k=0;
      p2=octal(n,c2);
      c2[p2]='\0';
      k=0;
      p3=hexadecimal(n,c3);
      c3[p3]='\0';
      k=0;
      printf("%d\t %s\t%s\t%s\n",n,c1,c2,c3);

    }
}
\end{verbatim}

\begin{center}
\begin{tabular}{rrrr}
23 & 10111 & 27 & 17\\
456 & 111001000 & 710 & 1C8\\
323 & 101000011 & 503 & 143\\
24533 & 101111111010101 & 57725 & 5FD5\\
432 & 110110000 & 660 & 1B0\\
35 & 100011 & 43 & 23\\
\end{tabular}
\end{center}
% Emacs 24.5.1 (Org mode 8.2.10)
\end{document}
