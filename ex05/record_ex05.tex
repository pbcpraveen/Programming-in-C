% Created 2018-04-15 Sun 14:08
\documentclass[11pt]{article}
\usepackage[utf8]{inputenc}
\usepackage[T1]{fontenc}
\usepackage{fixltx2e}
\usepackage{graphicx}
\usepackage{longtable}
\usepackage{float}
\usepackage{wrapfig}
\usepackage{rotating}
\usepackage[normalem]{ulem}
\usepackage{amsmath}
\usepackage{textcomp}
\usepackage{marvosym}
\usepackage{wasysym}
\usepackage{amssymb}
\usepackage{hyperref}
\tolerance=1000
\usepackage{palatino}
\usepackage[top=1in, bottom=1.25in, left=1.25in, right=1.25in]{geometry}
\usepackage{setspace}
\setcounter{secnumdepth}{1}
\author{PRAVEEN KUMAR R}
\date{}
\title{Exercise 5: Arrays}
\hypersetup{
  pdfkeywords={},
  pdfsubject={},
  pdfcreator={Emacs 24.5.1 (Org mode 8.2.10)}}
\begin{document}

\maketitle
\begin{center}
\begin{tabular}{lr}
Assignment & 5\\
Reg No & 312217104114\\
Name & PRAVEEN KUMAR R\\
Grade & \\
Date & 27-02-2018\\
\end{tabular}
\end{center}
\begin{export}
\linespread{1.2}
\end{export}
\section{Array visitor}
\label{sec-1}
\subsection*{\textbf{Polynomial evaluation}:}
\label{sec-1-1}
polynomial $a_{n-1}x^{n-1} + a_{n-2}x^{n-2} + \ldots + a_{1}x + a_{0}$
is represented as an array.
\texttt{a[0:n]} of its coefficients. Write a program to compute the value
of a polynomial using Horner's rule. The crux of the algorithm is:
\hfill (\emph{accumulator})
\linespread{1}
\begin{verbatim}
s = 0
for i = n-1 down to 0:
  s = s * x + a[i]
\end{verbatim}
\begin{verbatim}
#include<stdio.h>
int main()
{
  float a[100],x;
  int n;
  scanf("%d",&n);
  for(int i=0;i<n;i++)
    scanf("%f",&a[i]);
  scanf("%f",&x);
  float s=0;
  for(int i=n-1;i>=0;i--)
    s=s*x+a[i];
  printf("%f",s);
}
\end{verbatim}

\begin{verbatim}
547.0
\end{verbatim}

\linespread{1.2}
Compare the algorithm with the algorithm for summing the items of
an array.
\subsection*{\textbf{Binary search}:}
\label{sec-1-2}
We are given a sorted array of numbers. Define a
  function \texttt{binary\_search()} to search for a number in the given sorted 
  array.
\subsubsection*{Alogorithm}
\label{sec-1-2-1}
\begin{verbatim}
def binary_search(a,n,t):
   l=0
   h=n
   m=(l+n)//2;
   while(l!=n):
       if(a[m]==t):
           return m;
       elsif(a[m]>t):
           l=m+1
       else
           h=m
       m=(l+h)//2
   return n
\end{verbatim}
\subsubsection*{Specification}
\label{sec-1-2-2}
\linespread{1}
\begin{verbatim}
binary_search(a, n, target)
\end{verbatim}
\linespread{1.2}
that searches for \texttt{target} in \texttt{a[0:n]} using binary search
algorithm. Let the function return an index \texttt{i} such that \hfill
(\emph{search})
\begin{verbatim}
a[0:i] < target <= a[i:n]
\end{verbatim}
\begin{verbatim}
#include<stdio.h>
int binary_search(int a[],int n,int t)
{
  int beg=0,end=n-1;
  int mid=(beg+end)/2;
  while(beg<=end)
    {
      if(a[mid]==t)
	return mid;
      else if(a[mid]<t)
	beg=mid+1;
      else 
	end=mid;
      mid=(beg+end)/2;
    }
  return -1;
}
int main()
{
  int a[100],n,t;
  scanf("%d",&n);
  for(int i=0;i<n;i++)
    scanf("%d",&a[i]);
  scanf("%d",&t);
  int pos= binary_search(a,n,t);
  printf("%d",pos);
}
\end{verbatim}

\begin{verbatim}
5
\end{verbatim}

\section{Sorting}
\label{sec-2}
\subsection*{Selection sort}
\label{sec-2-1}
\subsubsection*{problem description}
\label{sec-2-1-1}
\textbf{Selection sort}: Selection sort is an algorithm for sorting an
  array of items, say \texttt{a[0:n]}. The idea of the algorithm is
  expressed below:
\linespread{1}
\begin{verbatim}
swap a[0], a[minimum(a,0,n)]
swap a[1], a[minimum(a,1,n)]
swap a[2], a[minimum(a,2,n)]
...
swap a[n-2], a[minimum(a,n-2,n)]
\end{verbatim}
which uses \texttt{minimum(a, i, n)} to find the minimum of a subarray
\texttt{a[i:n]}.
\begin{verbatim}
selection_sort (a, 0, n):
   for i = 0 to n-2:
      swap a[i], a[minimum(a, i, n)]
\end{verbatim}
\linespread{1.2}
Implement selection sort, using \texttt{minimum()} function. Note:
remember that when a function changes the items of an array
parameter, the changes are effected in the items of the actual
array argument also.

Test the function from \texttt{main()} for several lists of numbers. Each
test should read a list of numbers from stdin.
\subsubsection*{Program}
\label{sec-2-1-2}
\begin{verbatim}
#include<stdio.h>
int minimum(int a[], int low, int high)
{
  int m=low;
  for(int i=low+1;i<high;i++)
    if(a[i]<a[m])
      m=i;
  return m;
}
void selectionsort(int a[],int n)
{
  int temp,min;
  for(int i=0;i<n-1;i++)
    {
      min=minimum(a,i,n);
      temp=a[min];
      a[min]=a[i];
      a[i]=temp;
    }
}

int main()
{
  int a[100],n;
  for( n=0; scanf("%d",&a[n])!=EOF;n++);
  for(int i=0;i<n;i++)
    printf("%d%c",a[i],i<n-1?',':'\n');
  selectionsort(a,n);
  for(int i=0;i<n;i++)
    printf("%d%c",a[i],i<n-1?',':'\n');
  return 0;
}
\end{verbatim}
\subsubsection*{Test}
\label{sec-2-1-3}
\begin{itemize}
\item Input
\label{sec-2-1-3-1}
24 -990 378 378 63 1 43 -98 382 3846 26 -727 173 2847 
\item Output
\label{sec-2-1-3-2}
\begin{center}
\begin{tabular}{rrrrrrrrrrrrrr}
24 & -990 & 378 & 378 & 63 & 1 & 43 & -98 & 382 & 3846 & 26 & -727 & 173 & 2847\\
-990 & -727 & -98 & 1 & 24 & 26 & 43 & 63 & 173 & 378 & 378 & 382 & 2847 & 3846\\
\end{tabular}
\end{center}
\end{itemize}

\section{Polish National Flag (PNF)*:}
\label{sec-3}
In an array of items \texttt{a[low:high]},
each item is either positive or negative. Define a function
\texttt{partition(a, low, high)} that partitions the array into two
subarrays \texttt{a[low:i]} and \texttt{a[i:high]} such that all the negative
items of the array form \texttt{[low:i]}, and all the positive items form
\texttt{[i:high]}. Test the function from \texttt{main()}. Use several lists of
numbers for testing. (Note: We will use this algorithm for
implementing \texttt{quicksort()}.)

\subsection*{Specification:}
\label{sec-3-1}
The partition algorithm takes an array
\texttt{a[low:high]} as the input and returns an index \texttt{i} as
the output such that all the negative items of the array
form \texttt{[low:i]}, and all the positive items form
\texttt{[i:high]}.

\subsection*{Algorithm development}
\label{sec-3-2}
The iterative step:
After a few iterations, there are 3 subarrays, \texttt{[low:i]},
\texttt{[i:j]}, and \texttt{[j:n]}.

Initially, \ldots{}\\

The next item to be scanned is \texttt{[j]}.  There are two
cases: \texttt{[j] < 0} and \texttt{0 <} [j]=.

\subsection*{Algorithm}
\label{sec-3-3}
\begin{verbatim}
partition a, low, high:
   i, j = low, high
   while j != high:
      if a[j] < 0:
         swap a[i], a[j]
         i = i+1
      j = j + 1
\end{verbatim}

\linespread{1}
\subsection*{Program}
\label{sec-3-4}
\begin{verbatim}
#include <stdio.h>

int read_array (int a[]);
void print_array (int a[], int low, int high);
int partition (int a[], int n);
void swap (int a[], int i, int j);
int main ()
{
  int a[100];
  int n;
  int i;

  n = read_array(a);
  print_array (a, 0, n);
  i = partition (a, n);
  print_array (a, 0, i);
  print_array (a, i, n);  
}


int read_array (int a[])
{
  int i;
  for (i = 0; scanf ("%d", &a[i]) != EOF; i++)
    ;
  return i;
}

void print_array (int a[], int low, int high)
{
  int i;

  for (i = low; i < high; i++)
    printf ("%d,", a[i]);
  printf ("\n");
}

int partition (int a[], int n)
{
  int i, j;

  i = 0;
  j = 0;
  while (j != n) {
    if (a[j] < 0) {
      swap (a, i, j);
      i++;
    }
    j++;
  }
  return i;
}

void swap (int a[], int i, int j)
{
  int t = a[i];
  a[i] = a[j];
  a[j] = t;
}
\end{verbatim}
\subsection*{Test}
\label{sec-3-5}
\subsubsection*{Input}
\label{sec-3-5-1}
20  30 -80 -30  0  10  -40  -90  0  50 60  -50
\subsubsection*{Output}
\label{sec-3-5-2}
\begin{center}
\begin{tabular}{rrrrrrrrrrrrl}
20 & 30 & -80 & -30 & 0 & 10 & -40 & -90 & 0 & 50 & 60 & -50 & \\
-80 & -30 & -40 & -90 & -50 &  &  &  &  &  &  &  & \\
10 & 20 & 30 & 0 & 50 & 60 & 0 &  &  &  &  &  & \\
\end{tabular}
\end{center}

\section{Dutch National Flag (DNF):}
\label{sec-4}
DNF is similar to PNF, but partitions the
array \texttt{a[l:h]} into three subarrays \texttt{[l:i]}, \texttt{[i:j]} and
\texttt{[j:h]}. Each item of the array has one of the three
properties. Items having the same property should form one subarray
each.
\subsection*{Specification}
\label{sec-4-1}
2 functions \texttt{print(a[l:h])}, used to print the array, \texttt{dnf()}
which takes array \texttt{a[l:h]} and \texttt{c} as inputs and arrange the array
based on \texttt{c}.
\subsection*{Prototype}
\label{sec-4-2}
\begin{verbatim}
void print(char a[], int l, int h);
int dnf(char a[], int l, int h, char c);
\end{verbatim}
\subsection*{Program Design}
\label{sec-4-3}
The program contains 2 functions \texttt{print(char a[], int l, int h)},
which prints the array, \texttt{dnf(char a[], int l, int h, char c)}, which
returns the index upto which the array has been rearranged, and \texttt{main()}
which gets input from \texttt{stdin} and calls the functions.
\subsection*{Algorithm}
\label{sec-4-4}
\begin{verbatim}
def print(a[],l,h):
   for i in range(l,h):
      print(a[i])
def dnf(a[],l,h,c):
   i,p=l,l
   while i<h:
      if a[i]==c:
         a[i],a[p]=a[p],a[i]
         p+=1
      i+=1
\end{verbatim}
\subsection*{Program}
\label{sec-4-5}
\begin{verbatim}
#include <stdio.h>

int read_array (int a[]);
void print_array (int a[], int low, int high);
int partition (int a[],int low, int n,int k);
void swap (int a[], int i, int j);

int main ()
{
  int a[100];
  int n;
  int i,j;

  n = read_array(a);
  print_array (a, 0, n);
  i = partition (a,0, n,-1);
  j = partition (a,i, n,0);
  print_array (a, 0, i);
  print_array (a,i,j);
  print_array (a, j, n);  
}


int read_array (int a[])
{
  int i;
  for (i = 0; scanf ("%d", &a[i]) != EOF; i++)
    ;
  return i;
}

void print_array (int a[], int low, int high)
{
  int i;

  for (i = low; i < high; i++)
    printf ("%d,", a[i]);
  printf ("\n");
}

int partition (int a[], int low,int n,int k)
{
  int i, j;

  i = low;
  j = low;
if(k==-1)
 {
   while (j != n) {
    if (a[j] < 0) {
      swap (a, i, j);
      i++;
    }
    j++;
  }
}
else if(k==0)
{
while (j != n) {
    if (a[j] == 0) {
      swap (a, i, j);
      i++;
    }
    j++;
  }
}
  return i;
}

void swap (int a[], int i, int j)
{
  int t = a[i];
  a[i] = a[j];
  a[j] = t;
}
\end{verbatim}
\subsection*{Test}
\label{sec-4-6}
\subsubsection*{Input}
\label{sec-4-6-1}
23  -90    67   -65    0   0  83  0  -282  56  -473  0  372  -34  56 
\subsubsection*{Output}
\label{sec-4-6-2}
\begin{center}
\begin{tabular}{rrrrrrrrrrrrrrrl}
23 & -90 & 67 & -65 & 0 & 0 & 83 & 0 & -282 & 56 & -473 & 0 & 372 & -34 & 56 & \\
-90 & -65 & -282 & -473 & -34 &  &  &  &  &  &  &  &  &  &  & \\
0 & 0 & 0 & 0 &  &  &  &  &  &  &  &  &  &  &  & \\
56 & 23 & 83 & 372 & 67 & 56 &  &  &  &  &  &  &  &  &  & \\
\end{tabular}
\end{center}
% Emacs 24.5.1 (Org mode 8.2.10)
\end{document}
