% Created 2018-04-15 Sun 01:35
\documentclass[11pt]{article}
\usepackage[utf8]{inputenc}
\usepackage[T1]{fontenc}
\usepackage{fixltx2e}
\usepackage{graphicx}
\usepackage{longtable}
\usepackage{float}
\usepackage{wrapfig}
\usepackage{rotating}
\usepackage[normalem]{ulem}
\usepackage{amsmath}
\usepackage{textcomp}
\usepackage{marvosym}
\usepackage{wasysym}
\usepackage{amssymb}
\usepackage{hyperref}
\tolerance=1000
\usepackage{palatino}
\usepackage[top=1in,bottom=1.25in,left=1.25in,right=1.25in]{geometry}
\usepackage{setspace}
\setcounter{secnumdepth}{1}
\author{PRAVEEN KUMAR R}
\date{}
\title{Exercise 8: Arrays and strings}
\hypersetup{
  pdfkeywords={},
  pdfsubject={},
  pdfcreator={Emacs 24.5.1 (Org mode 8.2.10)}}
\begin{document}

\maketitle
\begin{export}
\linespread{1.2}
\end{export}
\begin{center}
\begin{tabular}{ll}
Assignment & 8\\
Reg No & 312217104114\\
Name & PRAVEEN KUMAR R\\
Grade & \\
Date & 24/03/2018\\
\end{tabular}
\end{center}
\linespread{1.5}

\section{Array of strings:}
\label{sec-1}
\begin{enumerate}
\item Declare an array of pointers to store strings.
\end{enumerate}
\begin{verbatim}
char* names[N];
\end{verbatim}
\begin{enumerate}
\item Declare an array of pointers to store strings and initialize them (maybe, with the names of the 12 months). Terminate the array with NULL pointer.
\end{enumerate}

\subsection*{Program design:}
\label{sec-1-1}
This program has only the main().

\subsection*{Program:}
\label{sec-1-2}
\begin{verbatim}
#include<stdio.h>
int main()
{
  char* names[]={"January",
		   "February",
		   "March",
		   "April",
		   "May",
		   "June",
		   "July",
		   "August",
		   "September",
		   "October",
		   "November",
		   "December",
		   NULL};
  int i=0;
  while(names[i])
    {
      printf("%s\n",names[i]);
      i++;
    }  
  return 0;
}
\end{verbatim}
\subsection*{Output}
\label{sec-1-3}
\begin{center}
\begin{tabular}{l}
January\\
February\\
March\\
April\\
May\\
June\\
July\\
August\\
September\\
October\\
November\\
December\\
\end{tabular}
\end{center}

\section{Passing an array of strings:}
\label{sec-2}
\begin{enumerate}
\item Define a function to count the number of strings in the pointer array.
\end{enumerate}
\begin{verbatim}
int strings_length (char* names[]) {
    for (int i = 0; names[i] != NULL; i++) ;
    return i; 
}
\end{verbatim}
\begin{enumerate}
\item Print an array of strings.
\end{enumerate}
\begin{verbatim}
void strings_print (char* names[], int n) { 
    for (int i = 0; i < n; i++)
        printf ("%s\n", entries[i]);
}
\end{verbatim}

\subsection*{Program design:}
\label{sec-2-1}
This program has three functions:
\begin{enumerate}
\item main()
\item strings$_{\text{print}}$(names[],n)
\item strings$_{\text{length}}$(names[])
\end{enumerate}

\subsection*{Program:}
\label{sec-2-2}
\begin{verbatim}
#include<stdio.h>
int strings_length(char* names[])
{
  int i;
  for(i=0;names[i]!=NULL;i++)
    ;
  return i;
}
void strings_print(char* names[],int n)
{
  for(int i=0;i<n;i++)
    printf("%s\n",names[i]);
}
int main()
{
  char* names[13]={"Jan",
		   "Feb",
		   "Mar",
		   "Apr",
		   "May",
		   "Jun",
		   "Jul",
		   "Aug",
		   "Sep",
		   "Oct",
		   "Nov",
		   "Dec",
		   NULL};
  int n=strings_length(names);
  strings_print(names,n);
  return 0;
}
\end{verbatim}
\subsection*{Output}
\label{sec-2-3}
\begin{center}
\begin{tabular}{l}
Jan\\
Feb\\
Mar\\
Apr\\
May\\
Jun\\
Jul\\
Aug\\
Sep\\
Oct\\
Nov\\
Dec\\
\end{tabular}
\end{center}

\section{Allocate memory for a string:}
\label{sec-3}
We want to clone a C-string. Function string$_{\text{clone}}$(s) takes a C-string as input and returns another C-string as the output. It allocates just enough memory for a new character array, using malloc() and copies s to the newly created character array, making it a C- string.
\begin{verbatim}
char *s = "In the beginning was the word."; 
char *t = string_clone (s);
char* string_clone (char s[])
{
  char *t = (char*) malloc (strlen(s)); 
  strcpy (t, s);
  return t;
}
\end{verbatim}
Function strings$_{\text{read}}$(names) reads a sequence of lines from stdin into an array of strings. Function strings$_{\text{print}}$(names, n) prints an array of strings on stdout. Function string$_{\text{clone}}$(s) creates a character array and copies s to it and returns a pointer to the array.
\begin{verbatim}
int strings_read (char* names[]) {
  char line[MAXLINE]; 
  int i;
  for (i = 0; fgets (line, MAXLINE, stdin) != NULL; i++) { int n = strlen (line);
    line[n-1] = ’\0’;
    names[i] = string_clone (line);
  }
  return i; 
}
char* string_clone (char s[]) {
  char *t = (char*) malloc (strlen(s));  
  strcpy (t, s);
  return t;
}
void strings_print (char* names[], int n) {
  for (int i = 0; i < n; i++) 
    printf ("%s\n", names[i]);
}
\end{verbatim}

\subsection*{Program design:}
\label{sec-3-1}
This program has 4 functions:
\begin{enumerate}
\item main()
\item string$_{\text{clone}}$(s[])
\item strings$_{\text{print}}$(names[],n)
\item strings$_{\text{read}}$(names[])
\end{enumerate}

\subsection*{Program:}
\label{sec-3-2}
\begin{verbatim}
#include<stdio.h>
#include<string.h>
#define N 100
#define MAXLINE 100
char* string_clone(char s[])
{
  char* t=(char*)malloc(strlen(s));
  strcpy(t,s);
  return t;
}
int strings_read(char* names[])
{
  char line[MAXLINE];
  int i;
  for(i=0;fgets(line,MAXLINE,stdin)!=NULL;i++)
    {
      int n=strlen(line);
      line[n-1]='\0';
      names[i]=string_clone(line);
    }
  return i;
}
void strings_print(char* names[],int n)
{
  for(int i=0;i<n;i++)
    printf("%s\n",names[i]);
}
int main()
{
  char* names[N];
  int n=strings_read(names);
  strings_print(names,n);
  return 0;
}
\end{verbatim}


\subsection*{Input}
\label{sec-3-3}
\begin{verbatim}
The Unix operating system was conceived and implemented in 1969,
at AT&T's Bell Laboratories in the United States by Ken Thompson, 
Dennis Ritchie, Douglas McIlroy, and Joe Ossanna. First released 
in 1971, Unix was written entirely in assembly language,
as was common practice at the time. Later, in a key 
pioneering approach in 1973, it was rewritten in the C programming 
language by Dennis Ritchie (with the exception of some hardware and I/O routines).
The availability of a high-level language implementation 
of Unix made its porting to different computer platforms easier
\end{verbatim}
\subsection*{Output}
\label{sec-3-4}
\begin{center}
\begin{tabular}{lllllllllllll}
The & Unix & operating & system & was & conceived & and & implemented & in & 1969, &  &  & \\
at & AT\&T's & Bell & Laboratories & in & the & United & States & by & Ken & Thompson, &  & \\
Dennis & Ritchie, & Douglas & McIlroy, & and & Joe & Ossanna. & First & released &  &  &  & \\
in & 1971, & Unix & was & written & entirely & in & assembly & language,A &  &  &  & \\
as & was & common & practice & at & the & time. & Later, & in & a & key &  & \\
pioneering & approach & in & 1973, & it & was & rewritten & in & the & C & programming &  & \\
language & by & Dennis & Ritchie & (with & the & exception & of & some & hardware & and & I/O & routines).\\
The & availability & of & a & high-level & language & implementation &  &  &  &  &  & \\
of & Unix & made & its & porting & to & different & computer & platforms & easier &  &  & \\
\end{tabular}
\end{center}
\section{Sort an array of strings:}
\label{sec-4}
You have written selection$_{\text{sort}}$() to sort an array of numbers. Do necessary changes in selection$_{\text{sort}}$() to sort an array of n lines in lexicographic order. The specification is
\begin{verbatim}
void selection_sort (char* names[], int low,int high)
\end{verbatim}
Test your sort function from main(). Use an array of strings, initialized along with declaration, in main().

\subsection*{Program design:}
\label{sec-4-1}
This program has 5 functions:
\begin{enumerate}
\item minimum(m[],low,high)
\item swap(m[],a,b)
\item string$_{\text{print}}$(names[],n)
\item selection$_{\text{sort}}$(m[],low,high)
\item main()
\end{enumerate}

\subsection*{Program:}
\label{sec-4-2}
\begin{verbatim}
#include<stdio.h>
#include<string.h>
#define N 100
#define MAXLINE 100
int minimum(char* m[],int low,int high)
{
  int i,minpos = low;
  for(i = low+1; i < high; i++)
    {
      if(strcmp(m[i], m[minpos])<0)
	{
	  minpos = i;
	}

    }
  return minpos;
}


void swap(char* m[],int a ,int b)
{
  char* t = m[a];
  m[a] = m[b];
  m[b] = t;
}


void string_print(char* names[],int n)
{
  for(int i=0;i<n;i++)
    {
      printf("%s\n  ",names[i]);
    }
}


void selection_sort(char* m[],int low,int high)
{
  int min;
  for(int i=low; i < high-1;i++)
    {
      min = minimum(m,i,high);
      swap(m,i,min);
    }
}

char* string_clone(char s[])
{
  char* t=(char*)malloc(strlen(s));
  strcpy(t,s);
  return t;
}
int strings_read(char* names[])
{
  char line[MAXLINE];
  int i;
  for(i=0;fgets(line,MAXLINE,stdin)!=NULL;i++)
    {
      int n=strlen(line);
      line[n-1]='\0';
      names[i]=string_clone(line);
    }
  return i;
}

int main()
{
  char* names[N];
  int n=strings_read(names);
  selection_sort(names,0,n);
  string_print(names,n);
  return 0;
}
\end{verbatim}
\subsection*{Test}
\label{sec-4-3}
\subsubsection*{Input}
\label{sec-4-3-1}
\begin{verbatim}
monica
chandler
ross
rachel
joey
pheobe
gunther
richard
james
\end{verbatim}

\subsubsection*{Output}
\label{sec-4-3-2}

\begin{center}
\begin{tabular}{l}
chandler\\
gunther\\
james\\
joey\\
monica\\
pheobe\\
rachel\\
richard\\
ross\\
\end{tabular}
\end{center}

\section{Read a sequence of strings (lines) from stdin:}
\label{sec-5}
Write a function strings$_{\text{read}}$(lines) to read a sequence of lines from stdin. It stores thelinesinanarrayofstringschar* lines[],and returns the count of lines as the result. After you read each line from stdin, allocate memory using string$_{\text{clone}}$() and store it as a string in char* lines[].
Test your function. Read the name list of your class from stdin. Sort it and print it. 

\subsection*{Program design:}
\label{sec-5-1}
This program has 4 functions:
\begin{enumerate}
\item string$_{\text{clone}}$(s[])
\item string$_{\text{read}}$(names[])
\item string$_{\text{print}}$(names[],n)
\item main()
\end{enumerate}

\subsection*{Program:}
\label{sec-5-2}
\begin{verbatim}
#include<stdio.h>
#include<string.h>
#include<stdlib.h>
#define N 100
#define MAXLINE 100
char* string_clone(char s[])
{
  char* t=(char*)malloc(strlen(s)+1);
  strcpy(t,s);
  return t;

}
int string_read(char* names[])
{
  char line[MAXLINE];
  int i;
  for (i=0;fgets(line,MAXLINE,stdin)!=NULL;i++)
    {
      int n=strlen(line);
      line[n-1]='\0';
      names[i]=string_clone(line);
    }
  return i;

}

void string_print(char* names[],int n)
{
  for(int i=0;i<n;i++)
    {
      printf("%s\n",names[i]);
    }
}
int main()
{
  char* names[N];
  int n=string_read(names);
  string_print(names,n);
  return 0;
}
\end{verbatim}


\subsection*{Test}
\label{sec-5-3}
\subsubsection*{Input}
\label{sec-5-3-1}

\begin{verbatim}
The Unix operating system was conceived and implemented in 1969,
at AT&T's Bell Laboratories in the United States by Ken Thompson, 
Dennis Ritchie, Douglas McIlroy, and Joe Ossanna. First released 
in 1971, Unix was written entirely in assembly language,
as was common practice at the time. Later, in a key 
pioneering approach in 1973, it was rewritten in the C programming 
language by Dennis Ritchie (with the exception of some hardware and I/O routines).
The availability of a high-level language implementation 
of Unix made its porting to different computer platforms easier
\end{verbatim}
\subsubsection*{Output}
\label{sec-5-3-2}
\begin{center}
\begin{tabular}{lllllllllllll}
The & Unix & operating & system & was & conceived & and & implemented & in & 1969, &  &  & \\
at & AT\&T's & Bell & Laboratories & in & the & United & States & by & Ken & Thompson, &  & \\
Dennis & Ritchie, & Douglas & McIlroy, & and & Joe & Ossanna. & First & released &  &  &  & \\
in & 1971, & Unix & was & written & entirely & in & assembly & language, &  &  &  & \\
as & was & common & practice & at & the & time. & Later, & in & a & key &  & \\
pioneering & approach & in & 1973, & it & was & rewritten & in & the & C & programming &  & \\
language & by & Dennis & Ritchie & (with & the & exception & of & some & hardware & and & I/O & routines).\\
The & availability & of & a & high-level & language & implementation &  &  &  &  &  & \\
of & Unix & made & its & porting & to & different & computer & platforms & easier &  &  & \\
\end{tabular}
\end{center}


\section{Sort an array of strings based on string length:}
\label{sec-6}
The strings are sorted according to their length so that shorter lines come before longer ones in the result.

\subsection*{Program design:}
\label{sec-6-1}
This program has 5 functions:
\begin{enumerate}
\item minimum(m[],low,high)
\item swap(m[],a,b)
\item string$_{\text{print}}$(names[],n)
\item selection$_{\text{sort}}$$_{\text{len}}$(m[],low,high)
\item main()
\end{enumerate}

\subsection*{Program:}
\label{sec-6-2}
\begin{verbatim}
#include<stdio.h>
#include<string.h>
#define N 100
#define MAXLINE 100
char* string_clone(char s[])
{
  char* t=(char*)malloc(strlen(s));
  strcpy(t,s);
  return t;
}
int strings_read(char* names[])
{
  char line[MAXLINE];
  int i;
  for(i=0;fgets(line,MAXLINE,stdin)!=NULL;i++)
    {
      int n=strlen(line);
      line[n-1]='\0';
      names[i]=string_clone(line);
    }
  return i;
}
int minimum(char* m[],int low,int high)
{
  int i,minpos=low;
  for(i=low+1;i<high;i++)
    {
      if(strlen(m[minpos])>strlen(m[i]))
	minpos=i;
    }
  return minpos;
}
void swap(char* m[],int a,int b)
{
  char* temp=m[a];
  m[a]=m[b];
  m[b]=temp;
}
void selection_sort_len(char* m[],int low,int high)
{
  int min;
  for(int i=low;i<high-1;i++)
    {
      min=minimum(m,i,high);
      swap(m,i,min);
    }
}
void strings_print(char* names[],int n)
{
  for(int i=0;i<n;i++)
    printf("%s\n",names[i]);
}
int main()
{
  char* names[100];
  int n=strings_read(names);
  selection_sort_len(names,0,n);
  strings_print(names,n);
  return 0;
}
\end{verbatim}
\subsection*{Test}
\label{sec-6-3}
\subsubsection*{Input}
\label{sec-6-3-1}
praveen kumar
lorents
albert
george
ram
ziva
\subsubsection*{Output}
\label{sec-6-3-2}
\begin{center}
\begin{tabular}{ll}
ram & \\
ziva & \\
albert & \\
george & \\
lorents & \\
praveen & kumar\\
\end{tabular}
\end{center}

\section{Search a string in an array of strings:}
\label{sec-7}
We wish to insert a new string into an array of sorted strings. First, we need to find the right position where the new strings has to be inserted. Do the needed changes in binary$_{\text{partition}}$() to find the right position of a target string in a sorted array of strings.

\subsection*{Program design:}
\label{sec-7-1}
This program has 3 functions:
\begin{enumerate}
\item main()
\item binary$_{\text{partition}}$(m[],n[],low,high)
\item string$_{\text{print}}$(names[],low,high)
\end{enumerate}

\subsection*{Program}
\label{sec-7-2}
\begin{verbatim}
#include<stdio.h>
#include<stdlib.h>
#include<string.h>
#define N 100
#define MAXLINE 100
char* string_clone(char s[])
{
  char* t=(char*)malloc(strlen(s));
  strcpy(t,s);
  return t;
}
int strings_read(char* names[])
{
  char line[MAXLINE];
  int i;
  for(i=0;fgets(line,MAXLINE,stdin)!=NULL;i++)
    {
      int n=strlen(line);
      line[n-1]='\0';
      names[i]=string_clone(line);
    }
  return i;
}
void string_print(char* names[],int low ,int high)
{
  for(int i=low;i<high;i++)
    {
      printf("%s,  ",names[i]);
    }
  printf("\n");
}

int binary_partition(char* m[],char n[],int low,int high)
{
  char* t;
  t=(char*)malloc(strlen(n));
  strcpy(t,n);
  int mid=(low+high)/2;
  while(low!=high){
    if(strcmp(t,m[mid])==0)
      return mid;
    else if(strcmp(t,m[mid])<0)
      {
	high=mid;
      }
    else
      {
	low=mid+1;
      }
    mid=(low+high)/2;
  }
  return high;
}

int main()
{
  char* names[N];
  int n=strings_read(names);
  string_print(names,0,n);
  int r=binary_partition(names,"mouse",0,n);
  printf("%d\n",r);
  return 0;
}
\end{verbatim}



\subsection*{Test}
\label{sec-7-3}
\subsubsection*{Input}
\label{sec-7-3-1}
\begin{verbatim}
apple
banana
car
dog
hockey
mouse
packets
probability
queue
time
watch
\end{verbatim}
\subsubsection*{Output}
\label{sec-7-3-2}
\begin{center}
\begin{tabular}{rlllllllllll}
apple, & banana, & car, & dog, & hockey, & mouse, & packets, & probability, & queue, & time, & watch, & ,\\
5 &  &  &  &  &  &  &  &  &  &  & \\
\end{tabular}
\end{center}

\section{Insert a target string in the right position in a sorted array of strings:}
\label{sec-8}
Using binary$_{\text{partition}}$() find the “right” position of a target string in an array of sorted strings. Write a function shift$_{\text{right}}$() to shift the strings to the right of the target’s position to make room for the target. Insert the target so that new array remains sorted.

\subsection*{Program design:}
\label{sec-8-1}
This program has 4 functions:
\begin{enumerate}
\item string$_{\text{print}}$(names[],low ,high)
\item binary$_{\text{partition}}$(m[],n[],low,high)
\item shift$_{\text{right}}$(names[],r,n,add[])
\item main()
\end{enumerate}

\subsection*{Program:}
\label{sec-8-2}
\linespread{1}
\begin{verbatim}
#include<stdio.h>
#include<stdlib.h>
#include<string.h>
#define N 100
#define MAXLINE 100
char* string_clone(char s[])
{
  char* t=(char*)malloc(strlen(s));
  strcpy(t,s);
  return t;
}
int strings_read(char* names[])
{
  char line[MAXLINE];
  int i;
  for(i=0;fgets(line,MAXLINE,stdin)!=NULL;i++)
    {
      int n=strlen(line);
	  line[n-1]='\0';
	  names[i]=string_clone(line);
    }
  return i;
}
void string_print(char* names[],int low ,int high)
{
  for(int i=low;i<high;i++)
    {
      printf("%s ,  ",names[i]);
    }
  printf("\n");
}
void insert(char* a[], char k[],int r,int* n)
{
  int i=*n-1;
  while(i>=r)
    {
      a[i+1]=a[i];
      i--;
    }
  a[r]=(char*)malloc(strlen(k)+1);
  strcpy(a[r],k);
  (*n)++;
}

int binary_partition(char* m[],char n[],int low,int high)
{
  char* t;
  t=(char*)malloc(strlen(n));
  strcpy(t,n);
  int mid=(low+high)/2;
  while(low!=high){
    if(strcmp(t,m[mid])<0)
      {
	high=mid;
      }
    else
      {
	low=mid+1;
      }
    mid=(low+high)/2;
  }
  return mid;
}

int main()
{
  char* names[N];
  int n=strings_read(names);
  string_print(names,0,n);
  int r=binary_partition(names,"scanner",0,n);
  insert(names,"scanner",r,&n);
  string_print(names,0,n);
  return 0;
}
\end{verbatim}

\begin{center}
\begin{tabular}{lllllllllllll}
apple & banana & car & dog & hockey & mouse & packets & probability & queue & time & watch &  & \\
apple & banana & car & dog & hockey & mouse & packets & probability & queue & scanner & time & watch & \\
\end{tabular}
\end{center}


\subsection*{Test}
\label{sec-8-3}
\subsubsection*{input}
\label{sec-8-3-1}
\begin{verbatim}
apple
banana
car
dog
hockey
mouse
packets
probability
queue
time
watch
\end{verbatim}
\subsubsection*{Output}
\label{sec-8-3-2}
\begin{center}
\begin{tabular}{lllllllllllll}
apple & banana & car & dog & hockey & mouse & packets & probability & queue & time & watch &  & \\
apple & banana & car & dog & hockey & mouse & packets & probability & queue & scanner & time & watch & \\
\end{tabular}
\end{center}
% Emacs 24.5.1 (Org mode 8.2.10)
\end{document}
