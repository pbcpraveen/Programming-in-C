% Created 2018-04-15 Sun 16:11
\documentclass[11pt]{article}
\usepackage[utf8]{inputenc}
\usepackage[T1]{fontenc}
\usepackage{fixltx2e}
\usepackage{graphicx}
\usepackage{longtable}
\usepackage{float}
\usepackage{wrapfig}
\usepackage{rotating}
\usepackage[normalem]{ulem}
\usepackage{amsmath}
\usepackage{textcomp}
\usepackage{marvosym}
\usepackage{wasysym}
\usepackage{amssymb}
\usepackage{hyperref}
\tolerance=1000
\usepackage{palatino}
\usepackage[top=1in, bottom=1.25in, left=1.25in, right=1.25in]{geometry}
\usepackage{setspace}
\author{PRAVEEN KUMAR R}
\date{9-13 April 2018 (Mon-Thurs)}
\title{Exercise 11: Files}
\hypersetup{
  pdfkeywords={},
  pdfsubject={},
  pdfcreator={Emacs 24.5.1 (Org mode 8.2.10)}}
\begin{document}

\maketitle
\linespread{1.2}
\linespread{1.5}
\begin{center}
\begin{tabular}{lr}
Assignment & 11\\
Reg No & 312217104114\\
Name & PRAVEEN KUMAR R\\
Grade & \\
Date & 13-04-2018\\
\end{tabular}
\end{center}


\section{Read from file}
\label{sec-1}
\begin{enumerate}
\item \texttt{telephone.in} is a file.  It is a sequence of
lines. Each line has two fields, separated by \texttt{|}
character. 
\begin{enumerate}
\item Number
\item Name
\end{enumerate}
\linespread{1}
\begin{verbatim}
|JYOTHISHMATHI C V|217104066|
|KAILASHWAR N|217104067|
|KANDAVEL A|217104068|
|KANISHQ S|217104069|
\end{verbatim}
\linespread{1.5}

\item Define \texttt{Entry} as a structure composed of \texttt{number} and
\texttt{name}. Define an array of pointers to \texttt{Entry}
structures.
\item Write a function \texttt{read\_telephones()} that reads the file
and converts each line to an \texttt{Entry} structure, and
stores them in the array of pointers to \texttt{Entry}
structures.
\item Write a function \texttt{print\_entries()} to display the entries
in the array to \texttt{stdout}.
\end{enumerate}
\begin{verbatim}
#include <stdio.h>
#include <string.h>
#include <stdlib.h>
#define MAXLEN 100
#define N      100

struct entry {
  int number;
  char name[MAXLEN];
};
typedef struct entry Entry;

int read_entries (Entry* e[]);
void print_entries(Entry* e[], int n);

int main ()
{
  Entry* telephones[N];
  int n; 
  n = read_entries (telephones);
  print_entries(telephones,n);
  return 0;
}
void print_entries(Entry* e[], int n)
{
  for(int i=0;i<n;i++)
    {
      printf("%s\t%d\n",e[i]->name,e[i]->number); 
    }
}

int read_entries (Entry* e[])
{
  FILE* fp;
  int i=0;
  char line[MAXLEN];
  char* name;
  char* number;
  fp = fopen ("telephone1.in", "r");
  for (i = 0; fgets(line, MAXLEN, fp) != NULL; i++) {
    name = strtok (line, "|");
    number = strtok (NULL, "|");
    e[i]=(Entry*)malloc(sizeof(Entry));
    e[i]->number=atoi(number);
    strcpy(e[i]->name,name);
  }
  fclose(fp);
  return i;
}
\end{verbatim}

\begin{center}
\begin{tabular}{lr}
JYOTHISHMATHI C V & 217104066\\
KAILASHWAR N & 217104067\\
KANDAVEL A & 217104068\\
KANISHQ S & 217104069\\
KARAN D & 217104070\\
KARTHIKEYAN R & 217104071\\
KARTHIK VISWANATH S & 217104072\\
KAVITHA A & 217104073\\
KAVYA J & 217104074\\
KEERTHIVASAN RAJAVADIVEL & 217104075\\
KEVIN J THELLY & 217104076\\
KISHORE S M & 217104077\\
KRIJESHAN G & 217104078\\
KRISHNAKANTH E & 217104079\\
KUMAR H & 217104080\\
LAKSHMI NARASIMHAN R & 217104081\\
LOKESH S & 217104082\\
MALAVIKA T & 217104083\\
MANISHA L & 217104084\\
MANO BALAJE S & 217104085\\
MITHUMARY C M & 217104086\\
MOHAMED MUSARAF P M & 217104087\\
MONIKA N & 217104088\\
MOURIESH S K & 217104089\\
MUSUNURU YASASWI & 217104090\\
NACHIAPPAN N N & 217104091\\
NAKUL KRISHNAN & 217104092\\
NANDA H KRISHNA & 217104093\\
NANDHINI R & 217104094\\
NARESH KUMAR R & 217104095\\
NAVEENA M & 217104096\\
NAVEEN NARAYANAN & 217104097\\
NIMISH S & 217104098\\
NITIN NIKAMANTH A B & 217104099\\
PAVILA V & 217104100\\
PAVITHRA N & 217104101\\
PAVYA S & 217104102\\
POOJA S (29.12.1999) & 217104103\\
POOJA S (11.06.2000) & 217104104\\
PRADEEP KUMAR B & 217104105\\
PRAGATHEESHWARI JAYASANKER & 217104106\\
PRAGNA REDDY N & 217104107\\
PRANATHY M S & 217104108\\
PRANAVI SHEKHAR & 217104109\\
PRANAV RAVEENDRAN & 217104110\\
PRANAV VIJAY & 217104111\\
PRATHEEP S & 217104112\\
\end{tabular}
\end{center}


\begin{center}
\begin{tabular}{lr}
PRATHISH E & 217104113\\
PRAVEEN KUMAR R & 217104114\\
PREETHI S (04.11.1999) & 217104115\\
PREETHI S (25.11.1999) & 217104116\\
PRIYA J & 217104117\\
PRIYADHARSHINI N & 217104118\\
RAGHUL P & 217104119\\
RAHUL V & 217104120\\
RAJESH R & 217104121\\
RAJESWARA RAJAN M & 217104122\\
RAKESH M & 217104123\\
RAKSHANAA R & 217104124\\
RAMKAUSHIK R & 217104125\\
RAMYA NIVASINI U S & 217104126\\
RANJANA S & 217104127\\
REENU RITA P S & 217104128\\
RESHMA RAMESH BABU & 217104129\\
RIYA RAJU & 217104130\\
\end{tabular}
\end{center}

\section{Search for an entry}
\label{sec-2}
\begin{enumerate}
\item Define a function \texttt{search\_number()} that searches for a given
number and prints the number and the name.
\end{enumerate}
\begin{verbatim}
int search_number(int p,Entry* e[],int n)
{
  for(int i=0;i<n;i++)
    {
      if(e[i]->number==p)
	{
	  return i;
	}
    }
 return n;
}
\end{verbatim}
\begin{enumerate}
\item Define a function \texttt{search\_name()} that searches for a given
name and prints the number and the name. You can search
for a substring using the library function \texttt{strstr()}.
\end{enumerate}
\begin{verbatim}
int search_name(char p[],Entry* e[],int n)
{
  for(int i=0;i<n;i++)
    {
      if(strcmpi(e[i]->name==p)==0)
	{
	  return i;
	}
    }
 return n;
}
\end{verbatim}

\section{Insert an entry}
\label{sec-3}
Write a function \texttt{insert\_entry()} that reads a name and number
from the user and adds it to the array. If the number
already exists, it should not be inserted.
\begin{verbatim}
int insert(Entry p,Entry* e[],int* n)
{
  int pos=search_number(p.number,e,*n);
  if(pos<*n)
    {
      e[*n]=(Entry*)malloc(sizeof(Entry));
      for(int i=(*n)-1;i>=pos;i--)
	{
	  e[i+1]=e[i];
	}
      e[pos]->number=p.number;
      strcpy(e[i]->name,p.name);
      (*n)++;
      return 1;
    }
  return 0;
}
\end{verbatim}

\section{Delete an entry}
\label{sec-4}
Write a function \texttt{delete\_entry()} that reads a number from
the user and deletes it from the array. When you delete an
entry, pack the array by moving the subsequent entries up.
\begin{verbatim}
int delete(int p,Entry* e[],int* n)
{
  int pos=search_number(p,e,*n);
  if(pos<(*n))
    {
      for(int i=pos;i<n-1;i++)
	{
	  e[i]=e[i+1];
	}
      (*n)--;
      return 1;
    }
  return 0;
}
\end{verbatim}

\section{Interactive loop}
\label{sec-5}
Write a loop that interacts with the user: It reads one of
the options from the user and performs the function.
\begin{verbatim}
q quit 
s number (search for a number)
f name (search for a name)
i number name (insert an entry (number, name))
d number (delete the entry with the number)
\end{verbatim}
When the program quits, the array of \texttt{Entry} structures
should be written to the \texttt{telephone.in} file.
\begin{verbatim}
#include <stdio.h>
 #include <string.h>
 #include <stdlib.h>

 #define MAXLEN 100
 #define N      100

 struct entry {
   int number;
   char name[MAXLEN];
 };
 typedef struct entry Entry;

 int read_entries (Entry* e[]);
 Entry* get_node (char* name, char* number);
 void print_entries (Entry* e[], int n);
 int write_entries (Entry* e[], int n);
 void print_entry(Entry* e[],int p);
 int search_number(int p,Entry* e[],int n);
 int insert(Entry p,Entry* e[],int* n);
 int delete(int p,Entry* e[],int* n);

 int main ()
 {
   Entry* telephones[N];
   int n;
   char line[MAXLEN];
   char choice;
   char name[MAXLEN];
   int number,res;
   n = read_entries (telephones);
   while (1)
     {
       printf ("? ");
       fgets (line, MAXLEN, stdin);
       choice = line[0];
       switch (choice)
	 {
	 case 's':
	   sscanf (line+1, "%d", &number);
	   int pos=search_number(number,telephones,n);
	   if(pos<n)
	      print_entry(telephones,pos);
	   else
	      printf("No such record\n");
	   break;
	 case 'i':
	   sscanf (line+1, "%d%s", &number, name);
	   Entry p;
	   p.number=number;
	   strcpy(p.name,name);
	   res= insert(p,telephones,&n);
	   if(res==0)
		printf("Record already exists\n");
	   break;
	 case 'd':
	   sscanf (line+1, "%d", &number);
	   res=delete(number,telephones,&n);
	   break;
	 case 'q':
	   write_entries(telephones, n);
	   print_entries(telephones,n);
	   return 0;
     default:
       printf("Invalid choice\n");
	 }
 }
 return 0;
 }
 void print_entry(Entry* e[],int p)
 {
   printf("%s\t %d\n",e[p]->name,e[p]->number);
 }
 void print_entries (Entry* e[], int n)
 {
   for (int i = 0; i < n; i++)
     printf ("%d,%s,%d\n", i, e[i]->name, e[i]->number);
 }

 int read_entries (Entry* e[])
 {
   FILE* fp;
   int i;
   char line[MAXLEN];
   char* name;
   char* number;

   fp = fopen ("telephone.in", "r");
   for (i = 0; fgets(line, MAXLEN, fp) != NULL; i++)
     {
       name = strtok (line, "|");
       number = strtok (NULL, "|");
       e[i] = get_node (name, number);
     }
   fclose(fp);
   return i;
 }

 Entry* get_node (char* name, char* number)
 {
   Entry* t = (Entry*) malloc (sizeof(Entry));
   strcpy(t->name, name);
   t->number = atoi(number);
   return t;
 }

 int write_entries (Entry* e[], int n)
 {
   int i;
   FILE* fp;

   fp = fopen ("telephone.in", "w");
   for (i = 0; i < n; i++)
     fprintf(fp, "|%s|%d|\n", e[i]->name, e[i]->number);
   fclose (fp);
 }
 int search_number(int p,Entry* e[],int n)
 {
   for(int i=0;i<n;i++)
     {
       if(e[i]->number==p)
	 {
	   return i;
	 }
     }
   return n;
 }
 int insert(Entry p,Entry* e[],int* n)
   {
     int pos=search_number(p.number,e,*n);
     if(pos==*n)
       {
	 e[*n]=(Entry*)malloc(sizeof(Entry));
	 e[pos]->number=p.number;
	 strcpy(e[pos]->name,p.name);
	 (*n)++;
	 return 1;
       }
     return 0;
   }
  int delete(int p,Entry* e[],int* n)
     {
       int pos=search_number(p,e,*n);
       if(pos<(*n))
	 {
	   for(int i=pos;i<(*n)-1;i++)
	     {
	       e[i]=e[i+1];
	     }
	   (*n)--;
	   return 1;
	 }
       return 0;
     }
\end{verbatim}
\subsection{Test}
\label{sec-5-1}
\subsubsection{Output}
\label{sec-5-1-1}
\begin{verbatim}
praveen@praveen:~/final/ex11$ gcc tel4.c
praveen@praveen:~/final/ex11$ ./a.out
? s 217104114
PRAVEEN KUMAR R	 217104114
? s 217104131
No such record
? i 217104131 JAYARAMAN
? i 217104093 NANDA
Record already exists
? s 217104131
JAYARAMAN	 217104131
? d 217104131
? s 217104131
No such record
? q
0, JYOTHISHMATHI C V ,217104066
1,KAILASHWAR N,217104067
2,KANDAVEL A,217104068
3,KANISHQ S,217104069
4,KARAN D,217104070
5,KARTHIKEYAN R,217104071
6,KARTHIK VISWANATH S,217104072
7,KAVITHA A,217104073
8,KAVYA J,217104074
9,KEERTHIVASAN RAJAVADIVEL,217104075
10,KEVIN J THELLY,217104076
11,KISHORE S M,217104077
12,KRIJESHAN G,217104078
13,KRISHNAKANTH E,217104079
14,KUMAR H,217104080
15,LAKSHMI NARASIMHAN R,217104081
16,LOKESH S,217104082
17,MALAVIKA T,217104083
18,MANISHA L,217104084
19,MANO BALAJE S,217104085
20,MITHUMARY C M,217104086
21,MOHAMED MUSARAF P M,217104087
22,MONIKA N,217104088
23,MOURIESH S K,217104089
24,MUSUNURU YASASWI,217104090
25,NACHIAPPAN N N,217104091
26,NAKUL KRISHNAN,217104092
27,NANDA H KRISHNA,217104093
28,NANDHINI R,217104094
29,NARESH KUMAR R,217104095
30,NAVEENA M,217104096
31,NAVEEN NARAYANAN,217104097
32,NIMISH S,217104098
33,NITIN NIKAMANTH A B,217104099
34,PAVILA V,217104100
35,PAVITHRA N,217104101
36,PAVYA S,217104102
37,POOJA S (29.12.1999),217104103
38,POOJA S (11.06.2000),217104104
39,PRADEEP KUMAR B,217104105
40,PRAGATHEESHWARI JAYASANKER,217104106
41,PRAGNA REDDY N,217104107
42,PRANATHY M S,217104108
43,PRANAVI SHEKHAR,217104109
44,PRANAV RAVEENDRAN,217104110
45,PRANAV VIJAY,217104111
46,PRATHEEP S,217104112
47,PRATHISH E,217104113
48,PRAVEEN KUMAR R,217104114
49,PREETHI S (04.11.1999),217104115
50,PREETHI S (25.11.1999),217104116
51,PRIYA J,217104117
52,PRIYADHARSHINI N,217104118
53,RAGHUL P,217104119
54,RAHUL V,217104120
55,RAJESH R,217104121
56,RAJESWARA RAJAN M,217104122
57,RAKESH M,217104123
58,RAKSHANAA R,217104124
59,RAMKAUSHIK R,217104125
60,RAMYA NIVASINI U S,217104126
61,RANJANA S,217104127
62,REENU RITA P S,217104128
63,RESHMA RAMESH BABU,217104129
64,RIYA RAJU,217104130
65,JAYARAMAN,204104131
\end{verbatim}
% Emacs 24.5.1 (Org mode 8.2.10)
\end{document}
