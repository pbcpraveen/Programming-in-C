% Created 2018-07-21 Sat 22:45
\documentclass[11pt]{article}
\usepackage[utf8]{inputenc}
\usepackage[T1]{fontenc}
\usepackage{fixltx2e}
\usepackage{graphicx}
\usepackage{longtable}
\usepackage{float}
\usepackage{wrapfig}
\usepackage{rotating}
\usepackage[normalem]{ulem}
\usepackage{amsmath}
\usepackage{textcomp}
\usepackage{marvosym}
\usepackage{wasysym}
\usepackage{amssymb}
\usepackage{hyperref}
\tolerance=1000
\usepackage{palatino}
\usepackage[top=1in,bottom=1.25in,left=1.25in,right=1.25in]{geometry}
\usepackage{setspace}
\setcounter{secnumdepth}{1}
\author{PRAVEEN KUMAR R}
\date{30/01/2018}
\title{Exercise 1: Simple Programs}
\hypersetup{
  pdfkeywords={},
  pdfsubject={},
  pdfcreator={Emacs 25.2.2 (Org mode 8.2.10)}}
\begin{document}

\maketitle
\begin{center}
\begin{tabular}{ll}
Assignment & 1\\
Reg No & 312217104114\\
Name & PRAVEEN KUMAR R\\
Grade & \\
Date & \\
\end{tabular}
\end{center}

\section{Home work}
\label{sec-1}

\subsection*{Objective}
\label{sec-1-1}
To gain some familiarity with variables, assignment, output
statement, control flow, functions, and arrays in C.

\subsection*{Output (printf) statements}
\label{sec-1-2}
Write a program to print this text, with the second and the fourth
lines indended, on the stdout.
\begin{verbatim}
The heights by great men reached and kept
   Were not attained by sudden flight,
But they, while their companions slept,
   Were toiling upward in the night.
\end{verbatim}

\begin{enumerate}
\item Create a new program file \texttt{longfellow.c} in emacs. Edit your
program.
\item Create a makefile to compile your source program file
\texttt{longfellow.c}, compile it to an executable program
\texttt{longfellow}.
\item If there are any errors as you compile, fix them.
\item List the errors which occurred.
\end{enumerate}
\subsubsection*{Program}
\label{sec-1-2-1}
\begin{verbatim}
#include<stdio.h>
int main()
{
  char a[100];
  int i=0;
  while(fgets(a,100,stdin)!=NULL)
    {
      if(i%2==1)
	printf("\t");
      printf("%s",a);
      i++;
    }

  return 0;
}
\end{verbatim}
\subsubsection*{Test}
\label{sec-1-2-2}
\begin{itemize}
\item \textbf{Input}
\end{itemize}
\begin{verbatim}
The heights by great men reached and kept
Were not attained by sudden flight,
But they, while their companions slept,
Were toiling upward in the night.
\end{verbatim}

\begin{itemize}
\item \textbf{Output}
\begin{center}
\begin{tabular}{llllllll}
The & heights & by & great & men & reached & and & kept\\
 & Were & not & attained & by & sudden & flight, & \\
But & they, & while & their & companions & slept, &  & \\
 & Were & toiling & upward & in & the & night. & \\
\end{tabular}
\end{center}
\end{itemize}

\subsection*{Minimum of three numbers}
\label{sec-1-3}
Write a program \texttt{min2.c} to read two numbers from stdin and print
the smallest of the two numbers.
\begin{enumerate}
\item Implement the functionality in \texttt{main()}.
\item Divide your program into \texttt{main()} and another function
\texttt{min2()}. Function \texttt{min2()} takes two numbers as inputs and
returns as output the minimum of the two inputs.
\item Design a function \texttt{min3()} that takes three numbers as inputs
and returns as output the minimum of the three inputs.
\item Design \texttt{min3()} in at least three different ways. Make one
comment, good or bad, about each of the designs.
\end{enumerate}
\subsubsection*{Program}
\label{sec-1-3-1}
\begin{verbatim}
#include<stdio.h>
int min2(int a,int b)
{
  return a>b?b:a;
}
int min3(int a,int b,int c)
{
  return a<min2(b,c)?a:min2(b,c);
}
int main()
{
  int a,b,c;
  scanf("%d%d%d",&a,&b,&c);
  printf("%d",min3(a,b,c));
  return 0;
}
\end{verbatim}
\subsubsection*{Test}
\label{sec-1-3-2}
\begin{itemize}
\item \textbf{Input}
\end{itemize}
\begin{verbatim}
34 56 76
\end{verbatim}

\begin{itemize}
\item \textbf{Output}
\end{itemize}

\begin{verbatim}
34
\end{verbatim}

\subsection*{Power $x^n$}
\label{sec-1-4}
Construct a program \texttt{power.c}.
\begin{enumerate}
\item A number $b$ raised to the power $m$, $b^m$, can be calculated
by cumulatively multiplying 1 by $b$, $m$ times. For example, if
$b = 2$ and $m = 5$, then the process for calculating $b^m =
      2^5$ proceeds as shown in the table below:
\begin{center}
\begin{tabular}{ll}
p & p * 2\\
\hline
1 & \\
2$^{\text{1}}$ & 1 * 2\\
2$^{\text{2}}$ & 2$^{\text{1}}$ * 2\\
2$^{\text{3}}$ & 2$^{\text{2}}$ * 2\\
2$^{\text{4}}$ & 2$^{\text{3}}$ * 2\\
2$^{\text{5}}$ & 2$^{\text{4}}$ * 2\\
\end{tabular}
\end{center}
Implement a program to calculate the power $b^m$. Hardcode $b$
and $m$ into your program (no need to read them from the
user). Print the output, for example, as \verb~2^5 = 32~.
\end{enumerate}
\subsubsection*{Program}
\label{sec-1-4-1}
\begin{verbatim}
#include<stdio.h>
int power(int a,int n);
int power(int a,int n)
{
  if(n==1)
    return a;
  if(n%2==0)
    {
      int p=power(a,n/2);
      return p*p;
    }
  else
    return a*power(a,n-1);
}
int main()
{
  int a=7,n=6;
  printf("%d^%d=%d",a,n,power(a,n));
  return 0;
}
\end{verbatim}

\subsubsection*{Test}
\label{sec-1-4-2}
\begin{center}
\begin{tabular}{l}
7$^{\text{6}}$=117649\\
\end{tabular}
\end{center}


\begin{enumerate}
\item Read $b$ and $m$ from the user (stdin) for the power
program. First, print a prompt message to the user:
\begin{verbatim}
Enter the base and the exponent:
\end{verbatim}

\item Define a function \texttt{power(x, n)} that raises $x$ to the power
$n$. It takes \texttt{x} and \texttt{n} as parameters and returns the power
$x^n$ as the result. Write the code for \texttt{power()} before
\texttt{main()}. Call \texttt{power()} from \texttt{main()}.
\item Write the code for \texttt{power()} \emph{after} \texttt{main()} and see the errors
reported. Fix it: Let the code for \texttt{power()} be after
\texttt{main()}. But write the prototype of \texttt{power()} before \texttt{main()}.
\item In \texttt{power(x, n)}, instead of using a variable to count the
number of iterations completed, use the parameter \texttt{n} to count
the number iterations left to terminate the loop.
\end{enumerate}
\subsubsection*{Program}
\label{sec-1-4-3}
\begin{verbatim}
#include<stdio.h>
 int power(int a,int n);
 int power(int a,int n)
 {
   if(n==1)
     return a;
   if(n%2==0)
     {
       int p=power(a,n/2);
       return p*p;
     }
   else
     return a*power(a,n-1);
 }
 int main()
 {
   int a,n;
   scanf("%d%d",&a,&n);
   printf("%d^%d=%d",a,n,power(a,n));
   return 0;
 }
\end{verbatim}
\subsubsection*{Test}
\label{sec-1-4-4}
\begin{center}
\begin{tabular}{l}
4$^{\text{6}}$=4096\\
\end{tabular}
\end{center}

\begin{enumerate}
\item List four idioms for repeating a loop $n$ times.
\end{enumerate}
\textbf{Example 1}
\begin{verbatim}
int i=0;]
while(i<n)
   i++;
\end{verbatim}
\textbf{Example 2}
\begin{verbatim}
for(int i=0;i<n;i++);
\end{verbatim}
\textbf{Example 3}
\begin{verbatim}
int i=n-1;
while(i>=0)
   i--;
\end{verbatim}
\textbf{Example 4}
\begin{verbatim}
for(int i=n-1;i>=0;i--);
\end{verbatim}
\begin{enumerate}
\item Pass a negative exponent to \texttt{power()}. What is the error that
occurs in the run time? Rename \texttt{power()} as \texttt{pos\_power()} and
write a function \texttt{power()} that works correctly for any integer
exponent, positive or negative.
\end{enumerate}
\subsubsection*{Program}
\label{sec-1-4-5}
\begin{verbatim}
#include<stdio.h>
 float power(float a,float n);
 float power(float a,float n)
 {
   if(n==1)
     return a;
   if((int)n%2==0)
     {
       int p=power(a,n/2);
       return p*p;
     }
   else
     return a*power(a,n-1);
 }
 int main()
 {
   float a,n,p;
   scanf("%f%f",&a,&n);
    p=power(a,n<0?-n:n);
   if(n<0)
     p=1/p;
   printf("%0.0f^%0.0f=%f",a,n,p);
   return 0;
 }
\end{verbatim}
\subsubsection*{Test}
\label{sec-1-4-6}
\begin{center}
\begin{tabular}{l}
3$^{\text{-5}}$=0.004115\\
\end{tabular}
\end{center}

\subsection*{Table of powers}
\label{sec-1-5}
Populate a table with powers $b^m$ for a given range of $m$.
\begin{enumerate}
\item Modify your program \texttt{main()} to print powers $b^m$ for a number
$b$ for powers from 0 to 20. Read \texttt{b} and \texttt{m} from the
user. Format the output as shown below.
\begin{verbatim}
2^0  =       1
2^1  =       2
2^2  =       4
2^3  =       8
2^4  =      16
...
2^20 = 1048576
\end{verbatim}
\end{enumerate}
\subsubsection*{Program}
\label{sec-1-5-1}
\begin{verbatim}
#include<stdio.h>
int main()
{
  long int a,n,p=1;
  scanf("%ld%ld",&a,&n);
  for(int i=0;i<=n;i++)
      {
	printf("%ld^%d=%ld\n",a,i,p);
	p=p*a;
      }
}
\end{verbatim}
\subsubsection*{Test}
\label{sec-1-5-2}
\begin{itemize}
\item \textbf{Input}
\end{itemize}
\begin{verbatim}
3 15
\end{verbatim}
\begin{itemize}
\item \textbf{Output}
\end{itemize}
\begin{center}
\begin{tabular}{l}
3$^{\text{0}}$=1\\
3$^{\text{1}}$=3\\
3$^{\text{2}}$=9\\
3$^{\text{3}}$=27\\
3$^{\text{4}}$=81\\
3$^{\text{5}}$=243\\
3$^{\text{6}}$=729\\
3$^{\text{7}}$=2187\\
3$^{\text{8}}$=6561\\
3$^{\text{9}}$=19683\\
3$^{\text{10}}$=59049\\
3$^{\text{11}}$=177147\\
3$^{\text{12}}$=531441\\
3$^{\text{13}}$=1594323\\
3$^{\text{14}}$=4782969\\
3$^{\text{15}}$=14348907\\
\end{tabular}
\end{center}

\begin{enumerate}
\item Declare an array of 100 numbers.  Store 10 numbers in the
array. Write a function \texttt{print\_array()} to print a
subarray. The function takes three parameters, the array name
\texttt{a}, the lower bound \texttt{low}, and the upper bound \texttt{high} of the
subarray. Remember that the upper bound is open --- the
subarray \texttt{a[low:high]} consists of \texttt{a[low] ... a[high-1]}
(\texttt{a[high]} is not a part of the subarray). Drive
\texttt{print\_array()} from \texttt{main()}.
\end{enumerate}
\subsubsection*{Program}
\label{sec-1-5-3}
\begin{verbatim}
#include<stdio.h>
void print_array(int a[], int low, int high);
void print_array(int a[], int low, int high)
{
  printf("{");
  for(int i=low;i<high;i++)
    printf("%d %c ",a[i],i<high-1?',':' ');
  printf("}");
}
int main()
{
  int a[100]={0,2,4,6,8,10,12,14,16,18};
  print_array(a,2,8);
}
\end{verbatim}
\subsubsection*{Test}
\label{sec-1-5-4}
\begin{center}
\begin{tabular}{lrrrrl}
\{4 & 6 & 8 & 10 & 12 & 14   \}\\
\end{tabular}
\end{center}

\begin{enumerate}
\item Fill the subarray \texttt{a[0]...a[20]} with $2^0$ to $2^{20}$. Print
it.
\end{enumerate}
\subsubsection*{Program}
\label{sec-1-5-5}
\begin{verbatim}
#include<stdio.h>
 int main()
 {
   long int a=2,n=20,p=1,m[100];
   for(int i=0;i<=n;i++)
     {
       m[i]=p;
       p=p*a;
     }
   for(int i=0;i<=n;i++)
       {
	 printf("%d\n",m[i]);
       }
 }
\end{verbatim}

\begin{enumerate}
\item Fill the subarray \texttt{a[0]...a[20]} with $2^{-10}$ to
$2^{10}$. Print it.
\end{enumerate}
\begin{verbatim}
#include<stdio.h>
 float power(float a,int n);
 float power(float a,int n)
 {
   if(n==0)
     return 1;
   if(n%2==0)
     {
      float p=power(a,n/2);
       return p*p;
     }
   else
     return a*power(a,n-1);
 }
 int main()
 {
   int k=0;
   float a=2,n=10,p,m[21];
   for(int i=-10;i<=n;i++)
     {
       p=power(a,(i<0?-i:i));
       if(i<0)
	 p=1/p;
       m[k++]=p;
     }
   for(int i=0;i<k;i++)
     printf("%f \n",m[i]);
   return 0;
 }
\end{verbatim}
\subsubsection*{Test}
\label{sec-1-5-6}
\begin{center}
\begin{tabular}{r}
0.000977\\
0.001953\\
0.003906\\
0.007812\\
0.015625\\
0.03125\\
0.0625\\
0.125\\
0.25\\
0.5\\
1.0\\
2.0\\
4.0\\
8.0\\
16.0\\
32.0\\
64.0\\
128.0\\
256.0\\
512.0\\
1024.0\\
\end{tabular}
\end{center}

\section{Questions}
\label{sec-2}
Answer the following questions.
\begin{enumerate}
\item How many lines are printed by \texttt{printf("Hello,\textbackslash{}nworld!\textbackslash{}nBye,\textbackslash{}nworld!")}?
\begin{enumerate}
\item 1
\item 2
\item 3
\item 4
\end{enumerate}
Answer:4
\item What is the output?
\begin{verbatim}
a = 5; b = 10;
a = b;
b = a;
printf ("a = %d, b = %d\n", a, b);
\end{verbatim}
Answer: a=10, b=10

\item What is the output? What does the code do?
\begin{verbatim}
a = 5; b = 10;
a = a+b;
b = a-b;
a = a-b;
printf ("a = %d, b = %d\n", a, b);
\end{verbatim}
Answer: a = 10, b = 5

\item What is the output? What does the code do?
\begin{verbatim}
a = 5; b = 10; c = 15;
t = a;
a = b;
b = c;
c = t;
printf ("a = %d, b = %d, c = %d\n", a, b, c);
\end{verbatim}
Answer: a=10, b=15, c=5

\item Translate the expression $d = \sqrt{b^2 - 4ac}$ into C statement.\\

Answer: d= sqrt(b*b-4*a*c);

\item Translate the expression $d = \sqrt{(x_1 - x_2)^2 + (y_1 -
     y_2)^2}$ into C statement.\\

Answer: d= sqrt(pow(x$_{\text{1}}$-x$_{\text{2}}$,2)+pow((y$_{\text{1}}$-y$_{\text{2}}$),2));

\item What is the output?
\begin{verbatim}
a = 5; b = 10;
m = a;
if (b < m)
   m = b;
printf ("%d\n", m);
\end{verbatim}
Answer: 5
\item What is the output?
\begin{verbatim}
mark = 40;
if (mark < 50)
  grade = 'E';
if (mark < 60)
  grade = 'D';
if (mark < 70)
  grade = 'C'
printf ("%c\n", grade);
\end{verbatim}
Answer: C
\item Trace the process generated by the loop
\begin{verbatim}
n = 5;
f = 1; i = 0;
while (i < n) {
  f = f * i;
  i = i + 1;
}
\end{verbatim}
Answer:
\begin{center}
\begin{tabular}{llll}
i & f & i+1 & f*i\\
 &  &  & \\
\end{tabular}
\end{center}
\item Write a loop (while statement) which will generate the process shown in the table.
\begin{center}
\begin{tabular}{rrll}
q & r & r - 5 & q + 1\\
0 & 22 &  & \\
1 & 17 & 22 - 5 & 0 + 1\\
2 & 12 & 17 - 5 & 1 + 1\\
3 & 7 & 12 - 5 & 2 + 1\\
4 & 2 & 7 - 2 & 3 + 1\\
 &  &  & \\
\end{tabular}
\end{center}
Answer:
\begin{verbatim}
r = 22;
q = 0;
while(r-5>0)
 {
    r=r-5;
    q=q+1;
 }
\end{verbatim}
\end{enumerate}
% Emacs 25.2.2 (Org mode 8.2.10)
\end{document}
